\section{Theoretical Background}

\subsection{Microservices Architecture (MSA)}
Microservices Architecture (MSA) represents a fundamental shift in how we build software, organizing applications not as single, monolithic giants, but as suites of small, autonomous services that work together \cite{awsMicroservices}. In a traditional monolith, everything, from user management to data processing, is tightly woven into one large codebase. While this makes starting a project easy, it often turns into a nightmare for scalability and fault tolerance as the application grows \cite{fowlerMicroservices}.

MSA takes a different approach. It treats the application as a collection of independent services, each running in its own process and communicating through lightweight channels, like HTTP APIs or messaging buses \cite{newmanBuilding}. Each service acts as an expert in its own specific business domain. For ``Plant Up!'', this style is a perfect fit. It allows us to build and scale distinct parts of the system independently. For example, the service that ingests thousands of sensor readings per minute can be scaled up during a ``wake-up'' event without disturbing the social feed service, which might be quiet at that moment.

The ``Plant Up!'' backend is organized around domain-specific schemas in Supabase, which act as the dedicated data stores for these microservices:
\begin{itemize}
    \item \texttt{user\_schema}: Handles everything related to who the user is: identity, authentication, and personal stats like streaks.
    \item \texttt{social\_media\_schema}: Manages the community aspects: the posts you see and the comments you write.
    \item \texttt{gamification}: Houses the logic that makes plant care fun: quests, experience points (XP), and rewards.
    \item \texttt{microcontroller\_schema}: A high-speed lane dedicated solely to catching raw sensor data from the IoT devices.
\end{itemize}

A clear example of this separation is how we store sensor readings. They live in their own isolated table, completely separate from the user data:

\begin{figure}[H]
    \centering
    \includegraphics[width=0.8\textwidth]{images/controllers_table.png}
    \caption{Controllers table in microcontroller\_schema}
    \label{fig:controllers_table}
\end{figure}

This strict boundary improves robustness. If the sensor reading service crashes, it doesn't take the gamification or social features down with it.

\subsection{Core Characteristics of Microservices in IoT}
Applying Microservices Architecture to the Internet of Things (IoT) brings a unique set of challenges. IoT systems are messy: they are asynchronous, event-driven, and have to deal with the unpredictable real world.

\subsubsection{Autonomy and Database per Service}
One of the golden rules of MSA is ``Database per Service''. This means services are decoupled not just in their code, but also in their data state \cite{richardsonDbPerService}. It prevents the dangerous situation where changing a database table for one service accidentally breaks another. In ``Plant Up!'', the Gamification Service keeps its own scorecard of player statistics, which is completely independent of the raw stream of incoming sensor data.

\begin{figure}[H]
    \centering
    \includegraphics[width=0.8\textwidth]{images/gamification_stats.png}
    \caption{Gamification player statistics}
    \label{fig:gamification_stats}
\end{figure}

\subsubsection{Scalability and Elasticity}
IoT workloads are notoriously ``bursty''. You might have silence for an hour, and then suddenly thousands of devices wake up to report their status. MSA allows us to handle this by scaling the ``Ingestion Service'' horizontally (adding more instances to share the load) without wasting resources on the ``Social Service'', which might not need the extra power.

\subsection{Internet of Things (IoT) Constraints and Hardware}
At the heart of ``Plant Up!'' is the ESP32-S3 microcontroller. It is a powerful System-on-Chip (SoC) from Espressif Systems \cite{esp32s3Datasheet}, but it is still bound by the harsh reality of battery life. The software protocols choices we make dictate how long the hardware stays awake, and consequently, how long the battery lasts.

\subsubsection{Power Consumption and Sleep Modes}
The ESP32-S3 has several power modes, each consuming energy at a vastly different rate. Understanding these is key to our architecture:
\begin{itemize}
    \item \textbf{Active Mode:} Everything is on: CPU, Wi-Fi radio, the works. The device burns between \textbf{160mA and 260mA} \cite{lastMinuteActive}. This is expensive; we want to spend as little time here as possible.
    \item \textbf{Modem Sleep:} The CPU is running, but the radio is off. Power drops to about 20mA-30mA \cite{lastMinuteActive}.
    \item \textbf{Deep Sleep:} This is where the device spends most of its life. The CPU, Wi-Fi, and RAM are powered down. Only the tiny Ultra-Low Power (ULP) coprocessor and the Real-Time Clock tick away. Consumption plummets to a mere \textbf{10\textmu A – 150\textmu A} \cite{esp32s3Datasheet}.
\end{itemize}

\subsubsection{The Wake-Up Tax}
Deep sleep saves a massive amount of energy, but it comes with a ``wake-up tax''. When the device wakes up, it has to re-initialize its Wi-Fi, find the access point, and ask for an IP address. This dance typically takes \textbf{1 to 3 seconds}, burning a lot of energy (~150mA average) before we even send a single byte of data \cite{adafruitWake}. The choice of protocol (MQTT vs. REST) determines how much heavier this tax becomes.

\subsection{Communication Protocols: MQTT vs. REST}
Choosing the right communication protocol is arguably the most critical decision for our data layer. It directly impacts energy efficiency, latency, and reliability.

\subsubsection{MQTT (Message Queuing Telemetry Transport)}
MQTT is a lightweight, binary messaging protocol designed specifically for networks that are unreliable or have limited bandwidth. It works on a publish-subscribe model.

\textbf{ Architecture and Decoupling: }
Unlike REST, which connects two points directly, MQTT uses a \textbf{Broker} in the middle. The sensor (Publisher) sends data to a topic (like \texttt{plantup/sensor/01}) without caring who is listening. The Broker handles the job of routing that message to anyone who subscribed (like the Ingestion Service). This decouples the devices in \textbf{Space} (they don't need to know each other's IP) and \textbf{Time} (messages can be queued if a service is down).

\begin{figure}[H]
    \centering
    \includegraphics[width=0.8\textwidth]{images/mqtt_converted.png}
    \caption{MQTT Publish/Subscribe Model: The Broker acts as a central hub, efficiently distributing messages from sensors to various services \cite{psiborg2024mqtt}.}
    \label{fig:mqtt_diagram}
\end{figure}

\textbf{Packet Structure:}
MQTT is binary, which means it cuts out the fluff. The fixed header is only \textbf{2 bytes}. Compare that to text-based HTTP headers, which can easily bloat to hundreds of bytes.

\textbf{Quality of Service (QoS):}
MQTT gives us three levels of delivery assurance:
\begin{itemize}
    \item \textbf{QoS 0 (At most once):} Fire-and-forget. It uses the least energy, but if the message is lost, it's gone for good.
    \item \textbf{QoS 1 (At least once):} Guarantees delivery by waiting for an acknowledgment (PUBACK). Ideal for critical data.
    \item \textbf{QoS 2 (Exactly once):} A heavy four-step handshake. Usually too much overhead for battery-powered devices.
\end{itemize}

\subsubsection{REST (Representational State Transfer)}
REST is the standard architectural style of the web, using standard HTTP methods. It is synchronous and resource-oriented.

\textbf{Request-Response Model:}
REST is strictly client-server. The client asks for something (GET) or sends something (POST), and waits until the server replies.

\begin{figure}[H]
    \centering
    \includegraphics[width=0.8\textwidth]{images/rest_converted.png}
    \caption{REST Request-Response Workflow: Stateless clients must open a new connection for every request, incurring significant overhead \cite{dreamfactory2024rest}.}
    \label{fig:rest_diagram}
\end{figure}

\textbf{Statelessness and Overhead:}
The server remembers nothing between requests. This means every single request has to carry all its baggage (authentication tokens, headers, etc). Worse, if the device sleeps between readings, it has to perform a full TCP three-way handshake and often a TLS handshake for \textit{every single data transmission}. This ``Connection Overhead'' makes REST a power hog for frequent, small data packets compared to a persistent MQTT session.

\subsection{Data Consistency and the CAP Theorem}
In distributed systems, the CAP theorem (Consistency, Availability, Partition Tolerance) tells us that we can't have it all. When a network failure happens, we have to choose two out of three \cite{brewerCap, jaiswalCAP}.

For ``Plant Up!'', \textbf{Partition Tolerance (P)} is non-negotiable. Wireless sensor networks are inherently unreliable. So, we have to choose between CP and AP:
\begin{itemize}
    \item \textbf{CP (Consistency + Partition Tolerance):} We refuse requests if we can't guarantee the data is perfectly up-to-date. This risks losing data during network glitches.
    \item \textbf{AP (Availability + Partition Tolerance):} We accept data and serve requests even if some nodes are slightly out of sync.
\end{itemize}

``Plant Up!'' firmly chooses an \textbf{AP design}, embracing \textbf{Eventual Consistency}. It is acceptable if a user sees a soil moisture value that is a few seconds old, as long as the app stays responsive and doesn't crash. The MQTT broker acts as a shock absorber, buffering messages during network blips and ensuring they eventually reach the microservices layer.
