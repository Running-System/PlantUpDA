\section{Experimental Evaluation of MQTT vs.\ REST}

\subsection{Introduction to the Experiment}
To evaluate the suitability of MQTT and REST for real-time IoT data synchronization in the ``Plant Up!'' system, a controlled experiment was conducted using an ESP32-S3 microcontroller as the edge node and the Supabase backend as the cloud microservice platform. Both communication methods were tested under identical conditions to quantify their performance with respect to latency, energy consumption, transmission overhead and delivery reliability.

The objective of this experiment is to determine which protocol better satisfies the system requirements for low-latency updates, minimal battery usage and stable operation under intermittent Wi-Fi conditions. This evaluation directly supports the research question regarding protocol efficiency in a microservices-based IoT architecture.

\subsection{Experimental Setup}
The experimental system consists of three components:
\begin{enumerate}
    \item \textbf{Edge Node:} An ESP32-S3 device equipped with sensors and configured to wake from deep sleep, collect environmental readings and transmit them using either REST or MQTT.
    \item \textbf{Network Environment:} A consumer-grade 2.4\,GHz Wi-Fi network representing typical home conditions, including variable latency and occasional packet loss.
    \item \textbf{Backend Services:} A Supabase instance running the domain-specific schemas defined for ``Plant Up!'', particularly the \texttt{microcontroller\_schema.Controllers} table which receives sensor data.
\end{enumerate}

Each transmission sends the same JSON payload generated from mock sensor values, matching the schema:

\begin{figure}[H]
    \centering
    \includegraphics[width=0.8\textwidth]{images/Payload used for both MQTT and REST experiments.png}
    \caption{Payload used for both MQTT and REST experiments}
    \label{fig:payload_structure}
\end{figure}

REST transmissions were executed using HTTPS POST requests to the Supabase endpoint, while MQTT transmissions were published to a broker using QoS 1. All tests were repeated 100 times for statistical significance.

\subsection{Methodology and Metrics}
To capture the performance characteristics of both protocols, measurements were collected for the following metrics:

\begin{itemize}
    \item \textbf{Latency:} Time from device wake-up to successful data commit in the backend.
    \item \textbf{Energy Consumption:} Average energy used during the wake-transmit-sleep cycle, measured indirectly via active radio time.
    \item \textbf{Payload Size:} Total number of bytes transmitted, including protocol overhead.
    \item \textbf{Reliability:} Percentage of successful transmissions under variable network quality.
\end{itemize}

Latency was measured using timestamp logs on both the ESP32-S3 and the cloud function writing to the \texttt{Controllers} table. Payload size was obtained from captured network packets. Reliability was determined by counting insertions successfully recorded:

\begin{figure}[H]
    \centering
    \includegraphics[width=0.8\textwidth]{images/Verification of received packets.png}
    \caption{Verification of received packets}
    \label{fig:verification_packets}
\end{figure}

Energy consumption was approximated using wake duration, as the ESP32's current draw during active transmission is well-documented.

\subsection{Experimental Results}
Table~\ref{tab:results} summarizes the averaged results for 100 transmissions per protocol.

\begin{figure}[H]
    \centering
    \includegraphics[width=0.8\textwidth]{images/markdown-table.png}
    \caption{Comparison of MQTT and REST performance under identical conditions}
    \label{fig:results_table_img}
\end{figure}

The results indicate that MQTT consistently outperforms REST in all IoT-relevant metrics: latency, reliability and energy usage. The lower payload size of MQTT contributes significantly to reduced wake duration.

\subsection{Discussion}
The experiment reveals clear trade-offs between MQTT and REST within the ``Plant Up!'' system. REST provides structured, authenticated communication suitable for user-driven actions but suffers from large header overhead, HTTP connection setup time and higher failure rates under unstable networks.

MQTT, by contrast, matches the constraints of the ESP32-S3. Its publish/subscribe model eliminates the need for repeated TLS handshakes, and its lightweight header structure minimizes energy consumption. Because the backend follows an eventually consistent microservices design, MQTT's asynchronous delivery semantics do not pose a disadvantage for plant monitoring data.

These findings align with the theoretical expectations established earlier in this chapter and support the use of MQTT as the primary ingestion mechanism for IoT sensor readings.

\subsection{Conclusion}
The experimental evaluation demonstrates that MQTT is significantly better suited than REST for transmitting real-time environmental data in a battery-powered IoT context. MQTT's reduced latency, lower overhead and higher reliability enable ``Plant Up!'' to maintain responsive and energy-efficient operation even under inconsistent network conditions.

REST remains valuable for mobile client interactions, authentication and structured queries, but MQTT should serve as the default protocol for sensor-to-cloud communication. This hybrid approach best satisfies the system requirements for robustness, efficiency and user experience.
