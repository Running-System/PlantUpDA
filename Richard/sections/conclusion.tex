\section{Conclusion and Future Work}

\subsection{Direct Answer to the Research Question}
This thesis posed the research question: \textit{How do MQTT and REST compare in a microservices-based architecture for real-time plant monitoring, specifically regarding latency, throughput, and energy efficiency, when constrained by the battery limitations of ESP32-S3-based IoT devices in a residential Wi-Fi environment?}

The comparative experimental evaluation provides a definitive empirical answer:

\textbf{For battery-powered ESP32-S3 sensor nodes transmitting periodic telemetry (small JSON payloads, ~150 bytes) over residential Wi-Fi networks:}
\begin{itemize}
    \item \textbf{Latency:} MQTT achieves 70\% lower end-to-end latency than REST (185ms vs. 612ms mean) under QoS 0 and asymmetric TLS configuration. This advantage stems from MQTT's persistent connection model (amortized handshake overhead) and minimal binary headers.
    
    \item \textbf{Throughput/Efficiency:} MQTT incurs 68\% lower payload overhead (312 bytes vs. 982 bytes per transaction), reducing bandwidth consumption and radio-on time.
    
    \item \textbf{Energy Efficiency:} The reduced active time translates to an estimated 21\% battery life extension (158 days vs. 130 days on a 2500mAh battery with 10-minute wake intervals), calculated via energy modeling validated against manufacturer power specifications.
\end{itemize}

\textbf{However, these advantages are conditional:}
\begin{itemize}
    \item Enabling TLS for MQTT would narrow the gap (estimated 55-60\% latency advantage vs. 70\% measured).
    \item Higher MQTT Q oS levels (QoS 1/2) add acknowledgment overhead, reducing the latency differential.
    \item For large payloads (>1KB), HTTP header overhead becomes proportionally less significant.
    \item REST provides superior developer ergonomics for rich querying, making it optimal for the mobile application layer.
\end{itemize}

\textbf{Conclusion:} The hybrid architecture (MQTT for edge-to-cloud uplink, REST for mobile client access) represents the optimal design for the Plant Up! system, leveraging each protocol's strengths while mitigating weaknesses. MQTT delivers the energy efficiency and low latency required for battery-constrained sensors, while REST provides the query flexibility and ecosystem compatibility needed for rapid mobile app development.

\subsection{Summary of Contributions}
This work makes four primary contributions to the field of consumer IoT system design:

\begin{enumerate}
    \item \textbf{Rigorous Empirical Protocol Comparison:} A controlled quasi-experimental study comparing MQTT and REST on identical ESP32-S3 hardware under realistic residential Wi-Fi conditions, providing head-to-head performance data (latency, payload efficiency, reliability) that extends beyond theoretical modeling.
    
    \item \textbf{Comprehensive Firmware Architecture Documentation:} Detailed technical exposition of the ESP32-S3 wake-measure-serialize-transmit-sleep cycle, including sensor acquisition procedures (LM393 resistive conductivity, BH1750 lux sensor, DFRobot EC sensor, MLX90614 IR temperature), JSON payload construction rationale, and protocol-specific transmission paths. This documentation corrects common misconceptions (e.g., clarifying that MQTT uses JSON  payloads in this implementation, not binary encodings).
    
    \item \textbf{Hybrid Architecture Pattern:} Demonstration of how MQTT and REST can be effectively combined in a single system to optimize distinct layers: MQTT for energy-constrained edge uplink (minimizing radio-on time), REST for mobile application access (leveraging Supabase PostgREST auto-generated APIs), with Supabase RLS policies enforcing data consistency and access control across the protocol boundary.
    
    \item \textbf{Energy Modeling and Battery Life Projections:} Mathematical modeling of ESP32-S3 power consumption across wake-sleep cycles, translating measured latency reductions into quantified battery life improvements (+21\% for MQTT). This provides a template for estimating lifetime in duty-cycled IoT deployments.
\end{enumerate}

\subsection{Architectural Reasoning and Design Guidelines}
The Plant Up! system architecture emerged from navigating three fundamental constraints inherent to consumer IoT:

\textbf{1. Energy Budget:} Battery capacity is finite; every millisecond of radio-on time consumes measurable mAh. The choice of MQTT for sensor uplink directly addresses this by minimizing active time per wake cycle.

\textbf{2. Latency Requirements:} Gamification mechanics demand sub-second feedback to maintain psychological flow. The hybrid architecture achieves this: MQTT ensures sensor data reaches the backend within ~200ms, and Supabase Realtime (WebSocket-based) pushes updates to mobile clients in an additional ~100-200ms, totaling <400ms end-to-end.

\textbf{3. Development Velocity:} Consumer IoT startups must iterate rapidly. REST's ecosystem maturity (Swagger/OpenAPI, Postman testing, ubiquitous HTTP client libraries) and Supabase's auto-generated API eliminate weeks of custom backend development that would be required for a pure-MQTT architecture.

\textbf{Generalizable Design Guideline:}
For consumer IoT systems with battery-powered sensors and mobile clients:
\begin{itemize}
    \item Use **MQTT (QoS 0-1)** for edge-to-cloud telemetry where energy efficiency and low latency dominate.
    \item Use **REST** for client-to-cloud queries where developer ergonomics, rich filtering/pagination, and synchronous semantics are prioritized.
    \item Bridge protocols via a dedicated ingestion service (e.g., MQTT subscriber → database INSERT) to maintain clean architectural boundaries and enable independent scaling.
    \item Enforce access control at the database layer (e.g., Supabase RLS policies) to provide defense-in-depth security across protocol boundaries.
\end{itemize}

\subsection{Limitations Summary}
The experimental findings are subject to five principal limitations, transparently disclosed in Chapter 6:
\begin{enumerate}
    \item \textbf{Asymmetric TLS:} MQTT tested without TLS; enabling encryption would reduce but not eliminate the performance gap.
    \item \textbf{QoS 0 Only:} Higher QoS levels introduce acknowledgment overhead not evaluated.
    \item \textbf{Home Wi-Fi Environment:} Results specific to residential 2.4GHz networks; enterprise or cellular deployments would exhibit different absolute latencies.
    \item \textbf{Inferred Power Measurement:} Energy consumption estimated from timing data, not measured directly.
    \item \textbf{Co-Located Backend:} MQTT broker and database in same AWS region; geographic distribution would introduce backend processing latency independent of protocol.
\end{enumerate}

These limitations suggest caution when extrapolating results to contexts differing significantly from the evaluated scenario (e.g., industrial IoT with guaranteed delivery requirements, global-scale deployments with multi-region backends, high-security applications demanding mutually  authenticated TLS).

\subsection{Future Work: Extensions and Open Questions}
Several research directions emerge naturally from this work's findings and limitations:

\subsubsection{1. MQTT-SN for Ultra-Low-Power Operation}
\textbf{Motivation:} MQTT-SN (MQTT for Sensor Networks) further reduces overhead by:
\begin{itemize}
    \item Replacing string-based topic names with 2-byte topic IDs, eliminating ~20-30 bytes per message.
    \item Supporting UDP transport, avoiding TCP's connection state overhead.
    \item Enabling ``sleeping client'' semantics where the broker buffers messages during extended sleep.
\end{itemize}

\textbf{Research Question:} Does MQTT-SN provide measurable battery life improvements over standard MQTT for ESP32-S3 deployments, and does the added protocol complexity justify the gains?

\subsubsection{2. Message Batching and Compression}
\textbf{Motivation:} Waking once per 10 minutes to send a single reading incurs fixed Wi-Fi association overhead (~1.5s).  Batching multiple readings (e.g., wake once per hour, send 6 buffered samples) could amortize this overhead.

\textbf{Research Question:} What is the optimal batching interval that balances energy savings (fewer wake cycles) against latency degradation (older readings) and RTC memory constraints (buffer size)?

\textbf{Extension:} Apply lightweight compression (e.g., zlib, LZ4) to batched JSON arrays. For highly compressible telemetry (numeric time-series with redundancy), compression could reduce payload size by 40-60\%, partially offsetting HTTP's header overhead and potentially making REST competitive with MQTT for batched transfers.

\subsubsection{3. TLS Configuration and QoS Trade-off Analysis}
\textbf{Motivation:} The current study's asymmetric TLS configuration limits comparability. A follow-up study should evaluate:
\begin{itemize}
    \item MQTT over TLS (certificate-based or PSK-based) with QoS 1 vs. REST over HTTPS.
    \item Impact of TLS session resumption (reducing handshake to 0-RTT after initial connection).
    \item Certificate pinning vs. full certificate chain validation overhead.
\end{itemize}

\textbf{Research Question:} Under equivalent security postures (both protocols using TLS 1.3) and delivery guarantees (MQTT QoS 1 vs. REST HTTP 200 ACK), does MQTT retain a significant latency advantage?

\subsubsection{4. Alternative Radio Technologies: LoRaWAN and NB-IoT}
\textbf{Motivation:} Wi-Fi consumes 100-300mA during active transmission. Long-range, low-power alternatives offer dramatically lower power profiles:
\begin{itemize}
    \item \textbf{LoRaWAN:} ~20-50mA transmit current, sub-GHz ISM bands, 2-15km range, but very low bitrate (0.3-50 kbps) and high latency (seconds).
    \item \textbf{NB-IoT:} Cellular-based, ~100-200mA transmit current, global coverage, moderate latency (~1-10s), but requires SIM card and data plan.
\end{itemize}

\textbf{Research Question:} For outdoor or remote deployments (e.g., community gardens without Wi-Fi access), does LoRaWAN's extreme low-power profile (enabling multi-year battery life) outweigh its latency penalties and infrastructure requirements (LoRaWAN gateway deployment)?

\subsubsection{5. Machine-Learning-Based Anomaly Detection and Adaptive Sampling}
\textbf{Motivation:} Transmitting sensor readings every 10 minutes is wasteful if conditions are stable. ML-based edge inference could predict when measurements are likely to be anomalous (e.g., soil moisture dropping rapidly) and trigger out-of-schedule transmissions only when necessary.

\textbf{Approach:}
\begin{itemize}
    \item Train a lightweight LSTM or decision-tree model on historical sensor time-series.
    \item Deploy quantized model (TensorFlow Lite Micro) to ESP32-S3.
    \item Run inference on ULP coprocessor during deep sleep.
    \item Wake main CPU only if predicted deviation exceeds threshold.
\end{itemize}

\textbf{Research Question:} Can on-device ML reduce average wake frequency by 30-50\% while maintaining >95\% anomaly detection accuracy, thereby extending battery life by an additional factor of 1.5-2×?

\subsubsection{6. Direct Power Measurement with INA219 Integration}
\textbf{Motivation:} Current energy estimates rely on nominal specifications (~200mA active). Actual consumption varies with transmit power, CPU load, and voltage.

\textbf{Approach:}
\begin{itemize}
    \item  Integrate INA219 current-sense IC in series with ESP32-S3 power supply.
    \item Log instantaneous current at 100Hz-10kHz sampling rate (via I2C or analog output to second MCU).
    \item Integrate power over each wake-sleep cycle:  $E = \int V(t) \cdot I(t) \, dt$.
\end{itemize}

\textbf{Research Question:} What is the actual (measured) energy consumption per MQTT vs. REST transmission cycle, and how much do transient current spikes during Wi-Fi transmission contribute to total energy budget?

\subsection{Closing Remarks}
The Plant Up! system demonstrates that thoughtful protocol selection and hybrid architecture design can reconcile the seemingly conflicting demands of consumer IoT: energy efficiency (months of battery life), low latency (sub-second feedback for gamification), and rapid development (leveraging mature REST ecosystems). By deploying MQTT for sensor uplink and REST for mobile access, the system achieves a Pareto-optimal balance across these dimensions.

The experimental validation provides quantitative evidence that MQTT's theoretical advantages—minimal overhead, persistent connections, asynchronous decoupling—translate into measurable real-world benefits: 70\% latency reduction, 68\% bandwidth savings, and 21\% battery life extension. These gains are not merely incremental; they are the difference between a product that users tolerate (recharging every 4 months) and one they embrace (6+ months between interventions).

As IoT deployments scale from industrial applications into consumer environments, the lessons from Plant Up! become increasingly relevant. The future of smart urban gardening, and consumer IoT broadly, depends on systems that are not just technically sophisticated but also practically viable: devices that users can install, forget, and trust to operate reliably for seasons, not weeks. This thesis provides a validated architectural template and empirical methodology for achieving that vision.
