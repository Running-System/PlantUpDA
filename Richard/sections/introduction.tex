\section{Introduction}

\subsection{Background and Context}
In recent years, the Internet of Things (IoT) has woven itself into the fabric of daily life at an extraordinary pace, fundamentally changing how we interact with our physical surroundings. This digital transformation is perhaps most visible in the agricultural sector, which is currently navigating a shift known as ``Agriculture 4.0''. We are seeing a move away from traditional, intuition-based farming toward precise, data-driven decision-making, all enabled by interconnected sensor networks \cite{leherAg40}. While industrial agriculture has already reaped significant efficiency gains from this revolution, a parallel and equally important trend is taking root in the consumer world: Smart Urban Gardening \cite{streetsolverSmartGarden}.

Driven by rapid global urbanization, shrinking living spaces, and a rising awareness of environmental sustainability, urban gardening has grown into a significant movement \cite{growdirectorUrbanAg}. This is more than just a hobbyist trend; it signals a shift toward the Social Internet of Things (SIoT) \cite{bpbSiot}. The SIoT paradigm expands our traditional understanding of IoT by suggesting that objects can establish social relationships, not just with other machines but with people, creating a truly collaborative ecosystem \cite{jungWeon2025}. In this new context, a plant is no longer just a passive biological entity. It becomes an active participant in a digital social network, capable of communicating its needs and status directly to its caretaker.

However, bringing these advanced concepts into the home using consumer-grade hardware creates a unique set of engineering hurdles \cite{kaaiotChallenges}. Unlike industrial systems, which often rely on stable power grids and dedicated infrastructure, consumer IoT devices for plant care must survive in resource-constrained environments. These devices often need to run for months on small batteries while staying connected to congested and unstable home Wi-Fi networks. This reality demands a rigorous and creative approach to optimizing both hardware and software strategies \cite{emnifyChallenges}.

\subsection{MMotivation for Smart Gardening and the Social Internet of Things}
The current wave of urban agriculture represents a convergence of technological capability and social necessity. As city populations continue to expand, access to fresh produce becomes both more critical and more challenging. Smart gardening systems address this gap by empowering individuals to cultivate food and ornamental plants in constrained urban environments with minimal prior expertise. The integration of real-time sensor feedback transforms gardening from an art based on experience into a science accessible to novices.

Moreover, the incorporation of social and gamification elements addresses a fundamental psychological barrier: the delayed gratification inherent in plant care. Traditional gardening requires patience and sustained attention over weeks or months before visible results emerge. By providing immediate digital feedback on watering actions, light exposure adjustments, and fertilization events, smart systems create a continuous reinforcement loop that maintains user engagement. This transformation aligns with established principles of behavioral psychology, where immediate rewards significantly strengthen habit formation \cite{spinifyGamification}.

The Social IoT framework further enhances this experience by enabling plants to participate in social networks. Users can share their achievements, compare plant health metrics with friends, and receive community-driven care recommendations. This social layer converts solitary plant maintenance into a collaborative activity, leveraging peer motivation and collective knowledge to improve outcomes.

\subsection{Case Study: The ``Plant Up!'' Project}
This thesis uses the ``Plant Up!'' project as a primary testbed to investigate these architectural challenges. ``Plant Up!'' is a comprehensive IoT application designed to gamify the experience of plant care, turning routine maintenance into an engaging activity \cite{smarticoGamification}. By providing immediate digital feedback on biological processes, the system bridges the gap between human perception and the actual physiological needs of a plant.

The system is built on a robust three-tier architecture that combines distributed sensor units, a scalable cloud microservices backend, and a user-facing mobile application \cite{cheThreeTier}. At the edge, ESP32-S3-based sensor nodes continuously monitor environmental metrics including soil moisture (via LM393 resistive conductivity sensor), ionic conductivity (DFRobot Gravity V2 EC sensor), ambient light intensity (BH1750 lux sensor with range up to approximately 65,000 lux), and surface temperature (MLX90614 infrared non-contact sensor). These nodes report to a cloud infrastructure powered by Supabase, which processes the raw telemetry, applying business logic to track both user progress and plant health. Finally, the application layer presents this data through a Flutter-based mobile interface that transforms mundane tasks into social achievements and rewards.

The system employs a hybrid communication architecture: sensor nodes transmit telemetry via MQTT to minimize power consumption and latency, while the mobile application consumes data via REST (specifically, Supabase's auto-generated PostgREST API). This dual-protocol approach allows the system to optimize for the distinct constraints of each subsystem while maintaining architectural coherence through a unified backend.

\subsection{Problem Statement}
Designing a real-time, socially integrated plant monitoring system reveals a fundamental conflict between three competing technical requirements: latency, energy efficiency, and data consistency. The`` Social'' aspect of the platform relies heavily on gamification mechanics that demand low latency to keep users immersed \cite{spinifyGamification}. For example, when a user waters their plant, they expect to see that action reflected in the app almost immediately, ideally within a few hundred milliseconds. Any perceptible delay disrupts the psychological feedback loop and diminishes user engagement.

However, to remain practical for home use, the hardware cannot drain its battery in a matter of days. It must utilize aggressive power-saving states, such as the ESP32-S3's deep sleep mode \cite{progElecDeepSleep}. In deep sleep, the main CPU, Wi-Fi radio, and the majority of RAM are powered down, reducing current consumption from approximately 160-260 mA to just 10-150 microamperes. Only the Real-Time Clock (RTC) memory and the Ultra-Low-Power (ULP) coprocessor remain active, enabling the device to wake on a timer or external trigger. While deep sleep dramatically extends battery life, waking up and reconnecting to Wi-Fi introduces unavoidable delays that directly clash with the need for real-time responsiveness \cite{redditEsp32Wake}.

Furthermore, the choice of communication protocol dictates the ``wake-up tax'', which is the cumulative energy expenditure required to establish a network connection before transmitting even a single byte of application data. Traditional web protocols like HTTP (REST) are robust and universal, but they impose significant overhead. Each HTTP request requires a full TCP three-way handshake, and when secured with TLS (as is standard practice), an additional TLS handshake phase is necessary. This sequence typically involves multiple round-trip messages, each consuming radio time and therefore battery capacity. Additionally, HTTP headers are verbose text-based structures that can exceed the payload size for small telemetry messages.

In contrast, lightweight protocols like MQTT are designed specifically to address these inefficiencies. MQTT employs a publish-subscribe model mediated by a broker, which decouples message producers from consumers. Once the initial connection to the broker is established, subsequent publishes require only minimal fixed-size headers (typically 2-5 bytes). However, it is critical to note that MQTT's performance gains are highly dependent on the Quality of Service (QoS) level, encryption configuration (TLS vs. plaintext), and the underlying network conditions. For instance, QoS 1 (at-least-once delivery) and QoS 2 (exactly-once delivery) introduce additional handshake steps that can partially negate the protocol's latency advantages \cite{psiborg2024mqtt, nabtoMqttRest}.

Finally, maintaining a consistent view of the system state in such a distributed architecture is a non-trivial challenge. As the CAP theorem (Consistency, Availability, Partition Tolerance) demonstrates, a distributed system cannot guarantee all three properties simultaneously during a network failure \cite{designGurusCAP}. In a residential IoT setting where network partitions are a common occurrence due to router reboots, Wi-Fi interference, and mobile client disconnections, the system must make calculated trade-offs. Prioritizing availability and partition tolerance (AP) typically implies adopting eventual consistency, where updates propagate asynchronously and temporary divergence between nodes is tolerated.

\subsection{Research Question}
To systematically address these challenges and identify the optimal architectural approach, this thesis poses the following primary research question:

\textit{How do MQTT and REST compare in a microservices-based architecture for real-time plant monitoring, specifically regarding latency, throughput, and energy efficiency, when constrained by the battery limitations of ESP32-S3-based IoT devices in a residential Wi-Fi environment?}

This question decomposes into several subsidiary inquiries:
\begin{itemize}
    \item What is the measurable difference in end-to-end latency (from sensor wake-up to server acknowledgment) between MQTT and REST under controlled home network conditions?
    \item How do the effective payload sizes (including protocol overhead) compare for typical telemetry messages?
    \item What are the implications of these differences for battery life in a device operating on a periodic wake-sleep cycle?
    \item Under what conditions (e.g., QoS level, payload size, network latency) does one protocol decisively outperform the other?
\end{itemize}

\subsection{Contributions}
This thesis makes the following contributions to the field of consumer IoT system design:
\begin{enumerate}
    \item \textbf{Empirical Protocol Comparison}: A rigorous, controlled empirical comparison of MQTT and REST protocols on identical ESP32-S3 hardware, measuring latency, payload efficiency, and reliability under realistic home Wi-Fi conditions. Unlike prior work that relies on simulation or theoretical analysis, this study provides real-world measurements.
    
    \item \textbf{Firmware Architecture Documentation}: A detailed technical description of the ESP32-S3 firmware architecture, including the wake → measure → serialize → transmit → sleep cycle, sensor acquisition procedures, and JSON payload formatting. This documentation clarifies common misconceptions (e.g., that MQTT uses binary payloads in this implementation) and provides a reference for similar projects.
    
    \ item \textbf{Hybrid Communication Architecture}: A demonstration of how MQTT and REST can be effectively combined in a single system to optimize for the distinct constraints of edge devices (low power, low latency) and mobile applications (rich querying, ease of development), while maintaining data consistency through Supabase Row-Level Security (RLS) policies.
    
    \item \textbf{Architectural Guidelines}: Practical design guidelines for consumer IoT deployments that balance real-time responsiveness, battery sustainability, and development complexity. These guidelines address protocol selection, QoS configuration, and consistency model trade-offs.
\end{enumerate}

\subsection{Thesis Structure}
The remainder of this thesis is organized as follows:

\textbf{Chapter 2} presents the theoretical foundations necessary to understand the subsequent analysis, including event-driven IoT architectures, microservices enablement through Supabase and PostgREST, energy modeling of ESP32-S3 deep-sleep modes, related work on consumer IoT constraints, and a deeper theoretical comparison of MQTT versus REST.

\textbf{Chapter 3} describes the complete system architecture of the Plant Up! platform, including detailed firmware architecture, sensor-specific technical corrections, the hybrid communication approach, and data consistency mechanisms.

\textbf{Chapter 4} details the experimental methodology, including research design (variables, hypotheses), hardware and software configuration, measurement instruments, repetition strategy, and validity measures.

\textbf{Chapter 5} presents the experimental evaluation results, including setup description, quantitative findings, interpretation, anomaly analysis, and discussion of network conditions.

\textbf{Chapter 6} discusses the implications of the results for real-world deployment, battery life consequences, experimental limitations (lack of TLS testing, QoS 0 only, home Wi-Fi variability, indirect power measurement), and comparison with related literature.

\textbf{Chapter 7} concludes the thesis by directly answering the research question, summarizing the architectural reasoning, and outlining future work including MQTT-SN, message batching, TLS evaluation, alternative radio technologies (LoRaWAN), machine-learning-based anomaly detection, and direct power measurement using INA219 or similar ICs.

\nocite{diplomarbeitSelf}
