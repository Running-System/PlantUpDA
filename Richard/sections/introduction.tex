\section{Introduction}

\subsection{Background and Context}
In recent years, the integration of the Internet of Things (IoT) into the fabric of daily life has accelerated at an unprecedented pace, fundamentally altering how humans interact with their physical environment. This digital transformation is particularly evident in the agricultural sector, which is currently undergoing a paradigm shift often referred to as ``Agriculture 4.0''. This fourth agricultural revolution is characterized by the replacement of traditional, heuristic-based farming methods with precise, data-driven decision-making processes facilitated by interconnected sensor networks \cite{leherAg40}. While widespread industrial adoption has already yielded significant efficiency gains in large-scale crop management, a parallel and equally demonstrable trend is emerging within the consumer sector: the rise of Smart Urban Gardening \cite{streetsolverSmartGarden}.

Driven by the compounding factors of rapid global urbanization, shrinking living spaces, and a growing societal consciousness regarding environmental sustainability, the urban gardening sector has experienced significant and sustained growth \cite{growdirectorUrbanAg}. This surge represents more than a mere hobbyist trend; it marks a fundamental shift toward the Social Internet of Things (SIoT) \cite{bpbSiot}. The SIoT paradigm extends the traditional definition of IoT by positing that objects are capable of establishing social relationships not only with other objects but also with humans, thereby fostering a collaborative ecosystem \cite{jungWeon2025}. In this context, a plant is no longer a passive biological entity but an active participant in a digital social network, communicating its needs and status to its caretaker.

However, translating these advanced theoretical concepts into utilizing consumer-grade hardware presents a unique set of architectural and engineering challenges \cite{kaaiotChallenges}. Unlike industrial systems, which often benefit from reliable power grids and dedicated communication infrastructure, consumer IoT devices designed for residential plant care must operate in resource-constrained environments. These devices are frequently required to function for months on limited battery reserves while communicating over congested and unstable residential Wi-Fi networks, necessitating a rigorous optimization of both hardware and software strategies \cite{emnifyChallenges}.

\subsection{Case Study: The ``Plant Up!'' Project}
This thesis utilizes the ``Plant Up!'' project as a primary testbed and case study to systematically investigate these architectural frictions. ``Plant Up!'' is a comprehensive IoT application designed to gamify and incentivize the experience of plant care \cite{smarticoGamification}. By providing immediate digital feedback on biological processes, the system aims to bridge the gap between human perception and plant physiology.

The system is constructed upon a robust three-tier architecture that combines distributed hardware sensor units, a scalable cloud microservices backend, and a user-facing mobile application \cite{cheThreeTier}. The edge layer consists of ESP32-based sensor nodes responsible for the continuous acquisition of environmental metrics, including soil moisture content, ambient temperature, relative humidity, and light intensity. These nodes communicate with a cloud infrastructure that processes the raw telemetry data, applying business logic to track user progress and plant health. Finally, the application layer presents this data to the user through an engaging interface that transforms routine maintenance tasks into social achievements.

\subsection{Problem Statement}
The architectural design of a real-time, socially-integrated plant monitoring system introduces a fundamental conflict between three competing technical requirements: latency, energy efficiency, and data consistency. The ``Social'' aspect of the platform relies heavily on gamification mechanics that require low-latency data transmission to maintain user immersion \cite{spinifyGamification}. For instance, a user expects immediate feedback when performing an action such as watering a plant.

However, to remain practical for home use without requiring frequent battery replacements, the hardware must utilize aggressive power-saving states, such as the ESP32's Deep Sleep mode \cite{progElecDeepSleep}. Research indicates that while deep sleep drastically extends battery life by shutting down the CPU and Wi-Fi radio, the subsequent wake-up and reconnection processes introduce unavoidable latency that directly conflicts with real-time requirements \cite{redditEsp32Wake}.

Furthermore, the choice of communication protocol dictates the ``wake-up tax''—the energy cost associated with establishing a network connection before any data can be transmitted \cite{redditEsp32Wake}. Traditional web protocols like HTTP (Hypertext Transfer Protocol) are robust and widely supported but carry significant header overhead and rely on verbose text-based formats. In contrast, lightweight protocols like MQTT (Message Queuing Telemetry Transport) are specifically designed to address these inefficiencies through binary payloads and publish/subscribe models \cite{psiborg2024mqtt}.

Additionally, maintaining a consistent view of the system state in such a distributed architecture is a non-trivial challenge. As the CAP theorem (Consistency, Availability, Partition Tolerance) dictates, a distributed system cannot simultaneously guarantee all three properties in the event of a network failure \cite{designGurusCAP}. In the context of a residential IoT system where network partitions are frequent, the system must make calculated trade-offs between data consistency and service availability.

\subsection{Research Question}
To systematically address these challenges and determine the optimal architectural approach, this thesis poses the following primary research question:

\textit{How do MQTT and REST compare in a microservices-based architecture for real-time plant monitoring, specifically regarding latency, throughput, and data consistency, when constrained by the energy limitations of battery-powered IoT devices \cite{nabtoMqttRest}?}
\nocite{diplomarbeitSelf}
