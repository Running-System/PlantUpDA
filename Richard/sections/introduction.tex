\section{Introduction}

\subsection{Background and Context}
The integration of the Internet of things (IoT) into our lives has been rapid, especially in the agricultural sector which has started a transformation known as ``Agriculture 4.0'', where data-driven decision making replaces traditional heuristic methods [1]. While industrial has benefited form that shift, a parallel trend is emerging in the consumer sector. Smart Urban gardening [2]. Driven by rapid urbanization and growing societal focus on sustainability, this sector is experiencing significant growth [3]. This surge represents a shift toward the Social Internet of Things (SIoT) [4]. In the SIoT paradigm, objects are capable of establishing social relationships with other objects and humans to foster collaboration[5] However, applying these advanced concepts to consumer grade hardware presents architectural challenges [6]. Unlike industrial systems, consumer IoT devices for plant care must operate in resource constrained environments, reyling on batteries and communicating over congested residential Wi-Fi [7].

\subsection{Case Study: The ``Plant Up!'' Project}
This thesis utilizes the ``Plant Up!'' project as a primary case study to investigate architecural frictions [8]. ``Plant Up!'' is an IoT application designed to gamify the experience of plant care [9]. The system combines hardware sensor units (IoT devices), cloud services and a mobile app to monitor soil moisture, temperature, humidity, light levels and other metrics in real time. The codebase relies on a three tier architecture: edge layer (esp32), the microservices layer and the Application layer [11].

\subsection{Problem Statement}
The design of a real time social plant monitoring system with gamification aspects introduces a conflict between latency, energy efficiency and data consistency [12]. The ``Social'' aspect relies on gamification mechanics that require low latency data transmission [13]. However, the hardware must utilize aggressive power saving states and techniques to be practical for home use [14]. Research indicates that while deep sleep extends battery life, the wake up and following processes cause latency that conflicts with real time requirements [15]. Furthermore, the choice of communication protocol dictates the ``wake up tax'' of the device [16]. Traditional web protocols like HTTP are robust but carry significant header overhead, whereas lightweight protocols like MQTT are designed for exactly that missing efficiency [17]. Additionally, maintaining a consistent view of the system state in a distributed architecture is non trivial, as the CAP theorem dictates trade offs between Consistency, Partition tolerance and Availability [18]

\subsection{Research Question}
To address these challenges, this thesis poses the following primary research question:
How do MQTT and REST compare in a microservices-based architecture for real-time plant monitoring, specifically regarding latency, throughput, and data consistency, when constrained by battery-powered IoT devices [19]?
