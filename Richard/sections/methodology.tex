\section{Experimental Methodology}

The comparative evaluation of MQTT versus REST in the ``Plant Up!'' system follows a controlled quasi-experimental design that isolates the impact of protocol choice on latency, payload efficiency, and inferred energy consumption. This chapter details the research design, experimental apparatus, measurement instruments, and validity considerations.

\subsection{Research Design and Hypotheses}
The experiment employs a within-subjects design where the same hardware apparatus is tested under both protocol conditions, minimizing confounding variables related to device variability, network infrastructure, or environmental conditions.

\subsubsection{Independent Variable}
The sole manipulated independent variable is the \textbf{communication protocol}, with two levels:
\begin{itemize}
    \item \textbf{MQTT}: Using the Eclipse Mosquitto open-source broker, QoS level 0 (at-most-once delivery), no TLS encryption, persistent TCP connection established during device wake.
    \item \textbf{REST}: Using Supabase PostgREST API over HTTPS (TLS 1.2/1.3), stateless request-response model, connection established per transmission cycle.
\end{itemize}

\subsubsection{Dependent Variables}
Three outcome variables are measured:
\begin{enumerate}
    \item \textbf{End-to-End Latency (ms):} Elapsed time from device boot (wake from deep sleep) to server acknowledgment receipt. This encompasses Wi-Fi association, DHCP negotiation, connection establishment, payload transmission, and acknowledgment.
    
    \item \textbf{Effective Payload Size (bytes):} Total bytes transmitted on the wire, including all protocol headers, measured via packet capture at the network interface. This accounts for TCP/IP overhead, application protocol headers (MQTT fixed header vs. HTTP headers), and the JSON payload itself.
    
    \item \textbf{Reliability (packet loss rate):} Proportion of transmission attempts that fail to receive server acknowledgment within a 10-second timeout window. This serves as a proxy for robustness under real-world network conditions.
\end{enumerate}

\subsubsection{Hypotheses}
Based on theoretical analysis presented in Chapter 2, the following null and alternative hypotheses are formulated:

\textbf{H1 (Latency):}
\begin{itemize}
    \item $H_{1,0}$: There is no significant difference in end-to-end latency between MQTT and REST protocols.
    \item $H_{1,A}$: MQTT exhibits significantly lower end-to-end latency than REST under identical network conditions.
\end{itemize}

\textbf{H2 (Payload Efficiency):}
\begin{itemize}
    \item $H_{2,0}$: MQTT and REST transmit equivalent byte quantities per message exchange.
    \item $H_{2,A}$: MQTT transmits significantly fewer bytes per exchange due to minimal header overhead.
\end{itemize}

\textbf{H3 (Energy Efficiency):}
\begin{itemize}
    \item $H_{3,0}$: The energy cost per transmission is equivalent for MQTT and REST.
    \item $H_{3,A}$: MQTT incurs lower energy cost per transmission due to reduced radio-on time (inferred from latency reduction).
\end{itemize}

Note: $H_3$ is evaluated indirectly via latency measurements, as radio-on time is the dominant contributor to active-mode energy consumption (~200mA constant draw).

\subsection{Hardware and Software Configuration}
To ensure reproducibility, the exact hardware and software configuration is documented in detail.

\subsubsection{Edge Device Specification}
\begin{itemize}
    \item \textbf{Microcontroller:} ESP32-S3-WROOM-1 module (Espressif Systems)
    \item \textbf{CPU:} Dual-core Xtensa LX7 @ 240 MHz
    \item \textbf{RAM:} 512 KB SRAM, 384 KB ROM
    \item \textbf{Flash:} 4 MB
    \item \textbf{Wi-Fi:} 802.11 b/g/n (2.4 GHz), configured for 802.11n with WPA2-PSK authentication
    \item \textbf{Transmit Power:} 19.5 dBm (typical)
    \item \textbf{Sensors Attached:} LM393 (soil moisture), BH1750 (light), DFRobot Gravity V2 (EC), MLX90614 (IR temperature). Note: For this experiment, sensors are powered but \textit{not} sampled; static dummy data is used to eliminate sensor acquisition time as a confounding variable.
    \item \textbf{Power Supply:} USB power (5V) to eliminate battery voltage sag effects; battery operation is not tested in this experiment.
\end{itemize}

\subsubsection{Firmware Specification}
\begin{itemize}
    \item \textbf{Framework:} ESP-IDF v5.2 (Espressif IoT Development Framework)
    \item \textbf{RTOS:} FreeRTOS 10.5.1
    \item \textbf{MQTT Library:} ESP-MQTT (native ESP-IDF component), configured for QoS 0, clean session, 60-second keep-alive
    \item \textbf{HTTP Library:} esp\_http\_client (native ESP-IDF component), configured for HTTP 1.1, TLS 1.2/1.3 via mbedTLS
    \item \textbf{JSON Library:} cJSON (included in ESP-IDF)
    \item \textbf{Timing Measurement:} High-resolution microsecond timer (\texttt{esp\_timer\_get\_time()}) with 1 $\mu$s resolution, used to timestamp key events: boot start, Wi-Fi connected, payload sent, acknowledgment received, sleep initiated.
\end{itemize}

Static JSON payload used for both protocols (eliminating serialization time variability):
\begin{verbatim}
{
  "device_id": "plant_test_01",
  "timestamp": "2025-12-09T15:00:00Z",
  "soil_moisture_pct": 45.0,
  "light_lux": 10000,
  "ec_us_cm": 800,
  "temperature_c": 21.5
}
\end{verbatim}

Payload size: 152 bytes (verified via \texttt{strlen()}).

\subsection{Network Infrastructure Configuration}
\begin{itemize}
    \item \textbf{Wi-Fi Router:} Consumer-grade 802.11n router (TP-Link Archer C7), 2.4 GHz band, channel 6 (chosen for minimal interference in test environment)
    \item \textbf{Network Topology:} ESP32-S3 → Wi-Fi router → Internet → Cloud services
    \item \textbf{RSSI Range:} Device positioned 3-5 meters from router, achieving RSSI between --65 dBm and --75 dBm (verified via \texttt{esp\_WiFi\_sta\_get\_rssi()})
    \item \textbf{MQTT Broker:} Eclipse Mosquitto 2.0.18, hosted on AWS EC2 t3.micro instance (1 vCPU, 1 GB RAM), located in eu-central-1 region
    \item \textbf{REST Endpoint:} Supabase Production instance, PostgreSQL 15, PostgREST API, located in eu-central-1 region (co-located with MQTT broker to minimize network latency asymmetry)
\end{itemize}

\subsection{Measurement Instruments and Data Collection}
Four complementary measurement techniques are employed to ensure data validity.

\subsubsection{On-Device Microsecond Timers}
The ESP32-S3 firmware instruments the wake-sleep cycle with high-precision timestamps at each phase transition:
\begin{enumerate}
    \item $T_0$ (Boot): Timestamp immediately upon \texttt{app\_main()} entry
    \item $T_1$ (Wi-Fi Connected): Timestamp in Wi-Fi event handler upon \texttt{IP\_EVENT\_STA\_GOT\_IP}
    \item $T_2$ (Payload Sent): Timestamp immediately after \texttt{mqtt\_publish()} or \texttt{http\_perform\_request()}
    \item $T_3$ (ACK Received): Timestamp upon MQTT PUBACK receipt or HTTP 200 OK response
    \item $T_4$ (Sleep Initiated): Timestamp immediately before \texttt{esp\_deep\_sleep\_start()}
\end{enumerate}

End-to-end latency is computed as $L = T_3 - T_0$. These timestamps are logged to a circular buffer in RAM and transmitted via UART to a connected PC for offline analysis (to avoid filesystem I/O affecting measurements).

\subsubsection{SQL Verification and Server-Side Logging}
After each transmission attempt, a verification query is executed against the Supabase database to confirm record insertion:
\begin{verbatim}
SELECT EXISTS(
  SELECT 1 FROM microcontroller_schema.Controllers
  WHERE device_id = 'plant_test_01'
    AND timestamp > (NOW() - INTERVAL '30 seconds')
);
\end{verbatim}

This provides ground truth for successful delivery. MQTT broker logs (Mosquitto \texttt{--verbose} mode) are separately archived to cross-reference PUBLISH receipt timestamps.

\subsubsection{Packet Capture with Wireshark}
Network traffic is captured at the Wi-Fi router using port mirroring to a laptop running Wireshark 4.0.3. Capture filters isolate traffic to/from the ESP32-S3 MAC address. For each transmission:
\begin{itemize}
    \item MQTT: Total bytes = Ethernet frame(s) containing: IP header (20 bytes), TCP header (20 bytes min), MQTT fixed header (2-5 bytes), topic string (~25 bytes), JSON payload (152 bytes).
    \item REST: Total bytes = Sum of all frames in HTTP transaction (TCP handshake, TLS handshake, HTTP request/response).
\end{itemize}

Post-processing scripts (Python with \texttt{scapy}) parse PCAP files to extract byte counts per transaction.

\subsubsection{Power Consumption Inference}
Direct power measurement (via INA219 or similar current-sense IC) is \textit{not} performed in this experiment due to hardware limitations. Instead, energy consumption is inferred from active-mode duration and nominal current draw:
\[
E_{estimated} = I_{active} \cdot (T_3 - T_1) \cdot V_{supply}
\]
where $I_{active} \approx 200$ mA (manufacturer specification), $V_{supply} = 3.3$ V. This provides a first-order approximation; actual consumption varies with transmit power and CPU load.

\subsection{Experimental Procedure and Repetition Strategy}
To achieve statistical power and account for network variability, each protocol condition undergoes 200 independent transmission cycles.

\subsubsection{Trial Procedure (Single Cycle)}
\begin{enumerate}
    \item Device resets via hardware button press or watchdog timer, clearing all RAM.
    \item Device boots, connects to Wi-Fi, transmits static payload via configured protocol (MQTT or REST).
    \item Device awaits acknowledgment with 10-second timeout.
    \item Upon acknowledgment (or timeout), device logs timestamps via UART, enters deep sleep for 5 seconds (artificially short for rapid data collection).
    \item Wake → Repeat.
\end{enumerate}

\subsubsection{Protocol Alternation and Randomization}
To mitigate temporal confounds (e.g., time-of-day network congestion), the 200 trials per protocol are interleaved rather than blocked:
\begin{itemize}
    \item Block randomization: Generate random sequence of 400 trials (200 MQTT, 200 REST).
    \item Execute trials sequentially per randomized order.
    \item Each trial completes before the next begins (no concurrent transmissions).
\end{itemize}

Total experiment duration: ~(400 cycles) × (average 20 seconds/cycle) = ~2.2 hours.

\subsubsection{Environmental Controls}
\begin{itemize}
    \item \textbf{Time Window:} All trials conducted between 14:00-17:00 local time to minimize diurnal network load variation.
    \item \textbf{Background Traffic:} Other devices on test network instructed to remain idle (no video streaming, large downloads, etc.).
    \item \textbf{Physical Environment:} Device and router positions fixed throughout experiment; no movement or obstructions introduced.
\end{itemize}

\subsection{Data Analysis Plan}
Collected data (device-side timestamps, packet captures, SQL verification logs) are merged into a unified dataset. For each trial, the record includes:
\begin{itemize}
    \item Protocol (MQTT or REST)
    \item End-to-end latency $L = T_3 - T_0$ (ms)
    \item Wi-Fi acquisition time $T_{WiFi} = T_1 - T_0$ (ms)
    \item Transmission time $T_{TX} = T_3 - T_1$ (ms)
    \item Effective payload size from PCAP (bytes)
    \item Delivery success (Boolean)
\end{itemize}

Descriptive statistics (mean, median, standard deviation, range) are computed per protocol. Hypothesis testing employs:
\begin{itemize}
    \item Independent samples t-test (or Mann-Whitney U if non-normal) for latency comparison.
    \item Effect size quantification via Cohen's d.
    \item Significance threshold $\alpha = 0.05$.
\end{itemize}

\subsection{Validity and Reproducibility Measures}
\subsubsection{Internal Validity}
\begin{itemize}
    \item \textbf{Control of Confounds:} Static payload, fixed network infrastructure, interleaved trial order, consistent environmental conditions.
    \item \textbf{Measurement Precision:} Microsecond-resolution timers minimize measurement error relative to millisecond-scale latencies.
\end{itemize}

\subsubsection{External Validity}
\begin{itemize}
    \item \textbf{Generalizability Limitations:} Results obtained in a single residential Wi-Fi environment may not generalize to enterprise networks, cellular (LTE/5G), or outdoor deployments.
    \item \textbf{Hardware Specificity:} ESP32-S3 results may not extend to other microcontrollers with different Wi-Fi subsystems or power profiles.
\end{itemize}

\subsubsection{Reproducibility}
To facilitate replication:
\begin{itemize}
    \item Complete source code (firmware, analysis scripts) version-controlled and publicly available.
    \item Exact hardware bill-of-materials (BOM) and part numbers documented.
    \item Raw experimental data (timestamped logs, PCAP files) archived and available upon request.
    \item Configuration files for MQTT broker and Supabase schema provided.
\end{itemize}

\subsection{Ethical Considerations and Limitations Disclosure}
\subsubsection{Limitations}
The experimental design incorporates several simplifications that constrain the interpretation of results:
\begin{enumerate}
    \item \textbf{No TLS for MQTT:} MQTT transmissions are unencrypted, whereas REST uses HTTPS (TLS). Enabling TLS for MQTT would increase handshake overhead and narrow the performance gap, but was omitted due to broker configuration complexity.
    
    \item \textbf{QoS 0 Only:} Only MQTT QoS 0 (fire-and-forget) is tested. Higher QoS levels (1 or 2) would introduce additional acknowledgment rounds, increasing latency and potentially altering conclusions.
    
    \item \textbf{Home Wi-Fi Variability:} The test network is a typical consumer environment prone to interference, signal fluctuations, and shared bandwidth. Results may not generalize to controlled laboratory or industrial settings.
    
    \item \textbf{Inferred Power Measurement:} Energy consumption is estimated from timing data and nominal current specifications, not measured directly with precision instrumentation (e.g., INA219 shunt monitor). This introduces uncertainty in energy comparisons.
    
    \item \textbf{Single Geographic Location:} Both MQTT broker and Supabase instance are co-located in eu-central-1. Results may differ if backend services are geographically distributed or exhibit higher network latency.
    
    \item \textbf{No Battery Operation:} Devices powered via USB at constant 5V. Battery voltage sag under load is not characterized, though this primarily affects transmit power stability.
\end{enumerate}

\subsubsection{Ethical Considerations}
No human subjects or personal data are involved in this experiment. All network communication occurs within controlled infrastructure (personal Wi-Fi, owned cloud accounts). No privacy or security risks are introduced.
