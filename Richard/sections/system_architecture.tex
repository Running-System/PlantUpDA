\section{Plant Up! System Architecture and Implementation}

\subsection{System Overview and Requirements}
The ``Plant Up!'' system is architected as a distributed Cyber-Physical Social System (CPSS), designed to bridge the gap between biological monitoring and social interaction. Unlike traditional agricultural data loggers which function as isolated silos, ``Plant Up!'' requires a continuous, bi-directional flow of data between the physical edge (the plant) and the social cloud (the community).

The architecture is stratified into three logical tiers, adhering to the standard IoT Reference Architecture:
\begin{itemize}
    \item \textbf{The Edge Tier (Perception Layer)}: Comprising ESP32-S3 microcontroller units responsible for sensing environmental variables (soil moisture, light, temperature) and executing actuations (water pumps).
    \item \textbf{The Platform Tier (Network \& Support Layer)}: A cloud-native microservices cluster hosted on AWS EKS (Elastic Kubernetes Service), responsible for data ingestion, processing, and storage.
    \item \textbf{The Application Tier (Social Layer)}: A mobile frontend that consumes processed data to drive gamification mechanics, such as ``Thirst Alerts'' and ``Garden Guild'' leaderboards.
\end{itemize}

The primary non-functional requirements driving this architecture are energy autonomy (battery life $>$ 30 days) and real-time responsiveness (latency $<$ 2 seconds for social interactions).

\subsection{Hardware Layer: The Edge Node}
The physical interface of the ``Plant Up!'' ecosystem is the sensor node, built around the ESP32-S3 System-on-Chip (SoC). This specific microcontroller was selected for its dual-core architecture, enabling dedicated handling of Wi-Fi stacks on one core while sensor logic runs on the other, and its Ultra-Low Power (ULP) co-processor, which is critical for the deep sleep cycles required by the project.

\subsubsection{Sensor Integration and Pinout}
The hardware design integrates three primary environmental sensors. To minimize corrosion and ensure long-term stability, a capacitive soil moisture sensor (v1.2) is used instead of a resistive one.

\begin{table}[h]
    \centering
    \begin{tabular}{|c|c|c|}
        \hline
        Sensor & Pin & Description \\
        \hline
        % Placeholder for table data
        Soil Moisture & GPIO XX & Capacitive Analog Input \\
        % Add more rows as needed based on missing source data
        \hline
    \end{tabular}
    \caption{Plant Up! Sensor Pin Configuration}
    \label{tab:pin_config}
\end{table}

\subsubsection{Firmware Implementation: The Deep Sleep Cycle}
To achieve the target battery life, the firmware implements a ``duty-cycled'' operation. The device spends approximately 99\% of its time in Deep Sleep mode, waking only to sample data and transmit. The following code snippet from the ``Plant Up!'' firmware demonstrates the deep sleep initialization logic, ensuring that on-board peripherals are powered down to reduce current leakage.

\begin{figure}[h]
    \centering
    % Placeholder for code snippet
    \begin{verbatim}
    // Code snippet placeholder: Deep Sleep Initialization
    // rtc_gpio_hold_en logic goes here
    \end{verbatim}
    \caption{Deep Sleep Logic Snippet}
\end{figure}

Note: The use of \texttt{rtc\_gpio\_hold\_en} is critical; without it, the floating pins could trigger sensor inputs or drain power through the voltage divider.

\subsection{Microservice Architecture Design}
The backend is composed of decoupled microservices deployed as Docker containers within a Kubernetes cluster. This architecture allows the ``Ingestion Service'' to scale independently during high-traffic events (e.g., a localized heatwave triggering simultaneous updates from thousands of devices).

\subsubsection{Service Boundaries}
The system is decomposed into the following core services:
\begin{itemize}
    \item \textbf{Ingestion Service}: The entry point for all telemetry. It acts as a protocol adapter, accepting MQTT packets or HTTP POST requests and normalizing them into an internal event format.
    \item \textbf{Plant Service}: Manages the botanical database, storing species-specific thresholds (e.g., ``Tomato'' needs 60-80\% moisture).
    \item \textbf{Social Service}: The gamification engine. It subscribes to telemetry events to calculate XP (Experience Points) and update leaderboards.
    \item \textbf{Notification Service}: Handles push notifications to the mobile app via Firebase Cloud Messaging (FCM).
\end{itemize}

\subsubsection{Infrastructure as Code (IaC)}
To ensure reproducibility, the infrastructure is defined using Kubernetes manifests. The following snippet shows the deployment configuration for the Ingestion Service, highlighting the resource limits imposed to simulate a constrained cloud environment for the thesis evaluation.

\begin{figure}[h]
    \centering
    % Placeholder for code snippet
    \begin{verbatim}
    # YAML Snippet Placeholder: Deployment Configuration
    \end{verbatim}
    \caption{Ingestion Service Deployment Manifest}
\end{figure}

This configuration ensures that the ingestion layer can handle concurrent connections from the MQTT broker without consuming excessive cluster resources.

\subsection{Real-Time Data Synchronization Mechanism}
The core conflict of this thesis—MQTT vs. REST—is implemented within the communication layer between the Edge Node and the Ingestion Service. The system is designed to support both protocols via a configuration flag in the firmware, allowing for direct A/B testing.

\subsubsection{Strategy A: MQTT Implementation}
In the MQTT mode, the device maintains a persistent TCP connection (when awake) and utilizes the Publish/Subscribe pattern. The topic structure is hierarchical, allowing the ``Social Service'' to subscribe to wildcards (e.g., \texttt{plantup/+/+/alert}) to detect issues across the entire user base efficiently.

Using \texttt{PubSubClient} allows for lightweight binary header transmission, theoretically reducing the ``radio-on'' time compared to HTTP.

\subsubsection{Strategy B: REST Implementation}
In the REST mode, the device opens a new TCP/TLS connection for every data transmission event. This follows a stateless request-response model.

The \texttt{HTTPClient} library abstracts the complexity, but the \texttt{http.begin()} and \texttt{http.end()} calls represent significant energy expenditure for the SSL handshake on every wake-up cycle.

\subsection{Social Gamification Logic Implementation}
The ``Social Service'' is the differentiating factor of ``Plant Up!''. It processes the raw telemetry to generate ``Gamified Events.'' For instance, if a user maintains their plant's moisture within the ideal range for 7 consecutive days, the service triggers a ``Streak Badge.''

This logic requires Data Consistency. If the MQTT broker drops a message, the user's streak might reset unfairly. The implementation uses an Eventual Consistency model where the ``Social Service'' acts as an idempotent consumer of the telemetry stream.

\begin{figure}[h]
    \centering
    % Placeholder for code snippet
    \begin{verbatim}
    // JS Snippet Placeholder: Backend Data Gap Handling
    \end{verbatim}
    \caption{Backend Idempotency Logic}
\end{figure}

This JavaScript snippet illustrates how the backend handles potential data gaps (network partitions) by checking for null values before penalizing the user, a necessary adaptation for the ``Availability over Consistency'' approach defined by the CAP theorem.
