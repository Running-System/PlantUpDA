\documentclass[a4paper, 12pt, twoside, openright]{report} 
%Einseitiges Layout
%\documentclass[a4paper, 12pt]{report} 


\raggedbottom
\usepackage{longtable}
\usepackage{tabularx}
\usepackage{lipsum}					    % Als Platzhalter für gewisse Stellen
\usepackage{ngerman}					%Deutsche Silbentrennung etc.
\usepackage[utf8]{inputenc}        	    % Umlaute in .tex Files normal schreibbar
\usepackage[T1]{fontenc}                    % Wichtig für korrekte Darstellung von Sonderzeichen und Copy-Paste

% Unter texnic-center alle Quelldateien unter Codierung ANSI abspeichern
% Auf Unixsystemen ISO 8859-15
\usepackage{helvet}						%Helvetic als Schriftart
\usepackage{courier}					%Courier als Schriftart für Listings
\usepackage{fancyhdr}					%Kopf- und Fußzeilen ändern
\usepackage{a4}							%A4 Randeinstellungen
\usepackage{makeidx}					%Indexkommandos
\usepackage{listings}					%Für zeilennummerierte Listings mit Hintergrund
\usepackage{color}					    %Für grauen Hintergrund in Listings
\usepackage{setspace}					%Größerer Zeilenabstand
\usepackage{graphicx}					%Grafiken einbinden
\usepackage{titlesec}
\titleformat*{\section}{\sffamily\Large\bfseries}
\titleformat*{\subsection}{\sffamily\large\bfseries}
\titleformat*{\subsubsection}{\sffamily\normalsize\bfseries}
\titleformat*{\paragraph}{\sffamily\normalsize\bfseries}
\titleformat*{\subparagraph}{\sffamily\normalsize\bfseries}
\usepackage{hyperref}
\usepackage{float}
\usepackage{pdfpages}					%Fremde pdfs einbinden
\usepackage[font={scriptsize}]{caption}
\usepackage{emptypage}
\usepackage{verbatim}                   %verbatim Jae
\usepackage{minted}                     %minted Jae
% \usepackage[final]{pdfpages}            %add pdfpages

\usepackage{booktabs}                  
\usepackage{caption}                   
\usepackage{geometry}       




%Dokumentationen zu den Paketen finden sich im Installationsordner 
%(normalerweise C:\Programme\texmf) unter docs und dort auch im Unterverzeichnis latex.

%Das Kompilieren des Dokuments benötigt bis zu 3 Durchläufe im alle Referenzen und
%Literatureinträge korrekt einzubinden.

%----------------------------------------------------------------------------------
% Listings 
%----------------------------------------------------------------------------------

%Definition des Aussehens der externen Listings
\def\source#1#2#3{     %  Sprache, Caption, Dateiname 
  %\global\advance\Sourcenummer by 1 
  %\index{Listing #1 #2}
  %\textbf{Listing-\the\Sourcenummer: #2}
  \lstinputlisting[language=#1,caption=#2]{#3} 
}

% Beispiele für eine Verwendung
%--------------------------------------------------------------------------------
% \source{xml}{\texttt{faces-config.xml}}{sources/faces-config.xml}
%
% \source{java}{\texttt{beans.UserBean.java}}{sources/jsf/user/UserBean.java}
%---------------------------------------------------------------------------------


%Definition des Aussehens der internen Listings
\definecolor{listinggray}{gray}{1.0}

%-------------------------------------------------------------------------
% Neues Listingformat
% --- kleine Schrift
% --- Keywords färbig
%-------------------------------------------------------------------------

\definecolor{dkgreen}{rgb}{0,0.6,0}
\definecolor{gray}{rgb}{0.5,0.5,0.5}
\definecolor{mauve}{rgb}{0.58,0,0.82}
\definecolor{orange}{RGB}{240, 105, 12}
\definecolor{green}{RGB}{156, 206, 43}

\lstset{ %
  language=Java,                  % the language of the code
  basicstyle=\footnotesize\ttfamily\bfseries,       % the size of the fonts that are used for the code
  numbers=left,                   % where to put the line-numbers
  numberstyle=\footnozesize,      % the size of the fonts that are used for the line-numbers
  stepnumber=1,                   % the step between two line-numbers. If it's 1, each line 
                                  % will be numbered
  numbersep=5pt,                  % how far the line-numbers are from the code
  backgroundcolor=\color{white},  % choose the background color. You must add \usepackage{color}
  showspaces=false,               % show spaces adding particular underscores
  showstringspaces=false,         % underline spaces within strings
  showtabs=false,                 % show tabs within strings adding particular underscores
  frame=single,                   % adds a frame around the code
  tabsize=2,                      % sets default tabsize to 2 spaces
  captionpos=b,                   % sets the caption-position to bottom
  breaklines=true,                % sets automatic line breaking
  breakatwhitespace=false,        % sets if automatic breaks should only happen at whitespace
  title=\lstname,                 % show the filename of files included with \lstinputlisting;
                                  % also try caption instead of title
  numberstyle=\tiny\color{gray},  % line number style
  keywordstyle=\color{blue},      % keyword style
  commentstyle=\color{dkgreen},   % comment style
  stringstyle=\color{mauve},      % string literal style
  escapeinside={\%*}{*)},         % if you want to add a comment within your code
  morekeywords={*,...}            % if you want to add more keywords to the set
}

\lstdefinelanguage{json}{
    basicstyle=\footnotesize\ttfamily,
    numbers=left,
    numberstyle=\tiny,
    stepnumber=1,
    numbersep=5pt,
    showstringspaces=false,
    breaklines=true,
    frame=single,
    backgroundcolor=\color{white},
    string=[s]{"}{"},
    comment=[l]{:\ "},
    morecomment=[l]{:"},
    literate=
     *{0}{{{\color{blue}0}}}{1}
      {1}{{{\color{blue}1}}}{1}
      {2}{{{\color{blue}2}}}{1}
      {3}{{{\color{blue}3}}}{1}
      {4}{{{\color{blue}4}}}{1}
      {5}{{{\color{blue}5}}}{1}
      {6}{{{\color{blue}6}}}{1}
      {7}{{{\color{blue}7}}}{1}
      {8}{{{\color{blue}8}}}{1}
      {9}{{{\color{blue}9}}}{1}
      {:}{{{\color{red}{:}}}}{1}
      {,}{{{\color{red}{,}}}}{1}
      {\{}{{{\color{black}{\{}}}}{1}
      {\}}{{{\color{black}{\}}}}}{1}
      {[}{{{\color{black}{[}}}}{1}
      {]}{{{\color{black}{]}}}}{1},
}


%Eigene Kommandos
\newcommand{\zb}{z.B.\ }				              %z.B.
\newcommand{\tm}{\texttrademark \ }			          %TM - Zeichen
\newcommand{\sk}[1]{\emph{siehe Kapitel \ref{#1}}}	  %siehe Kapitel <Referenz>
\newcommand{\lil}[1]{\emph{Listing \ref{#1}}}		  %Listing <Referenz>
\newcommand{\slil}[1]{\emph{siehe Listing \ref{#1}}}  %siehe Listing <Referenz>
\newcommand{\pr}{$\rightarrow\ $}			          %Pfeil nach rechts

%todo, maybe for composition api at new vue 3 features and vue apis


%-------------------------------------------------------------------------------
% Uebersichten i Anhang richtig formatieren
%-------------------------------------------------------------------------------

\makeatletter
\renewcommand*\l@section{\@dottedtocline{1}{0em}{2.8em}}
\renewcommand*\l@subsection{\@dottedtocline{2}{1.5em}{3.2em}}
\renewcommand*\l@subsubsection{\@dottedtocline{3}{3em}{4em}}
\renewcommand*\l@figure{\@dottedtocline{1}{1em}{2.5em}}
\renewcommand*\l@lstlisting{\@dottedtocline{1}{1em}{2.5em}}
\makeatother

%Es soll ein Index für diese Diplomarbeit erzeugt werden
%\makeindex

%Längen- und Absatzeinstellungen
\parindent=0pt		    %Kein Einrücken der ersten Zeile eines Absatzes
\parskip=12pt			%12pt Abstand zwischen 2 Absätzen
\doublespacing 	        %Doppelter Zeilenabstand
\onehalfspacing	        %Eineinhalbfacher Zeilenabstand

\setlength{\headheight}{15pt}		%Kopfzeile vergrößern (wegen 12pt Schriftgröße)	
\addtolength{\textwidth}{1.5cm}	%Rechten Rand verkleinern
\addtolength{\evensidemargin}{-1.5cm}


%--------------------------------------------------------------------------
% Vuecode minting Syntax highlighting
%--------------------------------------------------------------------------

\newminted[vuecode]{javascript}{
frame=lines,
framesep=2mm,
baselinestretch=1.2,
breaklines,
fontsize=\footnotesize,
linenos,
rulecolor=blue, %red
label=Code Example
}

%--------------------------------------------------------------------------
% HTMLcode minting Syntax highlighting
%--------------------------------------------------------------------------

\newminted[htmlcode]{javascript}{
frame=lines,
framesep=2mm,
baselinestretch=1.2,
breaklines,
fontsize=\footnotesize,
linenos,
rulecolor=blue, %red
label=Code Example
}

\newminted[javascriptcode]{javascript}{
frame=lines,
framesep=2mm,
baselinestretch=1.2,
breaklines,
fontsize=\footnotesize,
linenos,
rulecolor=blue, %red
label=Code Example
}

\newminted[csscode]{javascript}{
frame=lines,
framesep=2mm,
baselinestretch=1.2,
breaklines,
fontsize=\footnotesize,
linenos,
rulecolor=blue, %red
label=Code Example
}

\newminted[bashcode]{javascript}{
frame=lines,
framesep=2mm,
baselinestretch=1.2,
breaklines,
fontsize=\footnotesize,
linenos,
rulecolor=blue, %red
label=Code Example
}

%--------------------------------------------------------------------------
% Beginn Dokument
%--------------------------------------------------------------------------


\begin{document}
	\sffamily						                        %Schriftart setzen



    %\includepdf[pages=-]{pdf/druckensiebzig.pdf}
    %\phantomsection
    %\cleardoublepage
	\pagestyle{empty}
\singlespacing
\sffamily

\begin{flushleft}
	\includegraphics[scale=0.23]{images/Logo_transparent.png} \\
% 	\vspace{-2cm}
% 	\hspace{7cm}
% 	\Huge
% 	\textbf{- Diplomarbeit -} \\
% 	\hrulefill
\end{flushleft}
\vspace{-2.5cm}
\begin{flushright}
	\includegraphics[scale=0.10]{images/HTLzukunft.jpg} \\
% 	\vspace{-2cm}
% 	\hspace{7cm}
% 	\Huge
% 	\textbf{- Diplomarbeit -} \\
% 	\hrulefill
\end{flushright}

\vspace{-2.7cm}
\begin{center}
\textbf{HTL-Donaustadt}\\
\textbf{Höhere Abteilung für XYZ}
\end{center}
\vspace{-0.7cm}
\hrulefill

\begin{center}
\vspace{2cm}
\huge
\textbf{DIPLOMARBEIT}

\huge
\textbf{Titel der Diplomarbeit}

\large
\textbf{Untertitel der Diplomarbeit}
\end{center}

\begin{flushleft}
\large
\vspace{3cm}
\begin{small}
\begin{tabular}{lp{2cm}l}

\textbf{Ausgeführt im Schuljahr 2025/26 von:} &  & \textbf{Betreuer/Betreuerin:} \\
\\																
Abudi & & \\
Arun & & DI Lise Musterfrau \\
Dennis & & \\
Richard & & \\
\end{tabular}
\end{small}
\end{flushleft}

\vspace{1cm}

Wien, \today									%Externe .tex Datei für Titel einbinden
    \clearpage
    %Neue Seite beginnen

	
%-----------------------------------------------------------------
% Vorwort
%-----------------------------------------------------------------

    \cleardoublepage
 
	\pagestyle{plain}					    %Nur Fußzeile mit Seitennummer anzeigen lassen
	\pagenumbering{roman}					%Römische Nummerierung vor der eigentlichen Diplomarbeit
	\setcounter{page}{1}					%Bei 1 mit Nummerierung beginnen

    \cleardoublepage
    \addcontentsline{toc}{chapter}{Vorwort}
    \cleardoublepage
    
	\input{Allgemein/erklaerung}
    \cleardoublepage
    \addcontentsline{toc1}{section}{Erklärung}	%Erklärung händisch ins Inhaltsverzeichnis einfügen
	\cleardoublepage									%Neue Seite beginnen

	\phantomsection
	\addcontentsline{toc}{section}{Danksagungen}
	\begin{flushleft}
	\Large
	\textbf{Danksagungen\\}
	\vspace{1.5cm}
	
	\large
    Lorem ipsum dolor sit amet, consectetur adipiscing elit, 
    sed eiusmod tempor incidunt ut labore et dolore magna aliqua. 
    Ut enim ad minim veniam, quis nostrud exercitation ullamco 
    laboris nisi ut aliquid ex ea commodi consequat.
\end{flushleft}
	\cleardoublepage
	\phantomsection
	
	\addcontentsline{toc}{section}{Einleitung}
	\begin{flushleft}
	
	\subsection*{Einleitung}

	\subsubsection*{Subsubsection 1}
	
    Lorem ipsum dolor sit amet, consectetur adipiscing elit, sed eiusmod tempor incidunt ut labore et dolore magna aliqua. 
    Ut enim ad minim veniam, quis nostrud exercitation ullamco laboris nisi ut aliquid ex ea commodi consequat.
    
    \subsubsection*{Subsubsection 2}
	
    Lorem ipsum dolor sit amet, consectetur adipiscing elit, sed eiusmod tempor incidunt ut labore et dolore magna aliqua. 
    Ut enim ad minim veniam, quis nostrud exercitation ullamco laboris nisi ut aliquid ex ea commodi consequat.
    

\end{flushleft}
	\cleardoublepage
	\phantomsection
	
	\addcontentsline{toc}{section}{Abstract}
	\begin{flushleft}
	
	\subsection*{Abstract}
	%\vspace{1.5cm}

    Quis aute iure reprehenderit in voluptate velit esse cillum dolore eu 
    fugiat nulla pariatur. Excepteur sint obcaecat cupiditat 
    non proident, sunt in culpa qui officia deserunt mollit anim id est laborum.

	\subsubsection*{Subsubsection 1}
	
    Quis aute iure reprehenderit in voluptate velit esse cillum dolore eu fugiat nulla pariatur. Excepteur sint obcaecat cupiditat non proident, 
    sunt in culpa qui officia deserunt mollit anim id est laborum.

    \subsubsection*{Subsubsection 2}

    Quis aute iure reprehenderit in voluptate velit esse cillum dolore eu fugiat nulla pariatur. Excepteur sint obcaecat cupiditat non proident, 
    sunt in culpa qui officia deserunt mollit anim id est laborum. 
\end{flushleft}
	\cleardoublepage
	\phantomsection

    %gendererklärung
    \addcontentsline{toc}{section}{Gendererklärung}
	\begin{flushleft}
	
	\subsection*{Gendererklärung}
	\vspace{1cm}
	
    Das in dieser Arbeit gewählte generische Maskulinum bezieht sich zugleich auf die männliche, die weibliche und andere Geschlechteridentitäten. Zur besseren Lesbarkeit wird auf die Verwendung männlicher und weiblicher Sprachformen verzichtet. Alle Geschlechteridentitäten werden ausdrücklich mitgemeint, soweit die Aussagen dies erfordern.
\end{flushleft}
	\cleardoublepage
	\phantomsection
	
	\normalsize

	\markright{INHALTSVERZEICHNIS}
	\addcontentsline{toc}{chapter}{Inhaltsverzeichnis}
	\tableofcontents
	\cleardoublepage
	\phantomsection
 \thispagestyle{empty}
	
%-----------------------------------------------------------------
% Kopfzeilen definieren
%-----------------------------------------------------------------
	\pagestyle{fancyplain}

	\renewcommand{\sectionmark}[1]{\markright{\thesection\ #1}}
	\renewcommand{\chaptermark}[1]{\markright{\thechapter\ #1}}
	\lhead[\fancyplain{}{\sffamily\sl\thepage}]{\fancyplain{}{\sffamily\sl\rightmark}}
	\rhead[\fancyplain{}{\sffamily\sl\rightmark}]{\fancyplain{}{\sffamily\sl\thepage}}
	\cfoot{}

%--------------------------------------------------------------------------
% Kapitel einfügen / Sections einfügen
%--------------------------------------------------------------------------

\begin{flushleft}
\chapter{Einführung}
    %\fancyhead[l]{}
	\pagenumbering{arabic}	%Seiten wieder normal nummerieren
	\setcounter{page}{1}		%Bei 1 beginnen

    \textbf{Dennis Part}

\end{flushleft}

\chapter[Edge vs. Cloud Processing in Smart Gardening]{ Edge vs. Cloud Processing in Smart Gardening: A Comparative Study for Environmental Sensor Data}
\pagenumbering{arabic}
\setcounter{page}{1}

% \section{Introduction}

\subsection{Background and Context}
The integration of the Internet of things (IoT) into our lives has been rapid, especially in the agricultural sector which has started a transformation known as ``Agriculture 4.0'', where data-driven decision making replaces traditional heuristic methods [1]. While industrial has benefited form that shift, a parallel trend is emerging in the consumer sector. Smart Urban gardening [2]. Driven by rapid urbanization and growing societal focus on sustainability, this sector is experiencing significant growth [3]. This surge represents a shift toward the Social Internet of Things (SIoT) [4]. In the SIoT paradigm, objects are capable of establishing social relationships with other objects and humans to foster collaboration[5] However, applying these advanced concepts to consumer grade hardware presents architectural challenges [6]. Unlike industrial systems, consumer IoT devices for plant care must operate in resource constrained environments, reyling on batteries and communicating over congested residential Wi-Fi [7].

\subsection{Case Study: The ``Plant Up!'' Project}
This thesis utilizes the ``Plant Up!'' project as a primary case study to investigate architecural frictions [8]. ``Plant Up!'' is an IoT application designed to gamify the experience of plant care [9]. The system combines hardware sensor units (IoT devices), cloud services and a mobile app to monitor soil moisture, temperature, humidity, light levels and other metrics in real time. The codebase relies on a three tier architecture: edge layer (esp32), the microservices layer and the Application layer [11].

\subsection{Problem Statement}
The design of a real time social plant monitoring system with gamification aspects introduces a conflict between latency, energy efficiency and data consistency [12]. The ``Social'' aspect relies on gamification mechanics that require low latency data transmission [13]. However, the hardware must utilize aggressive power saving states and techniques to be practical for home use [14]. Research indicates that while deep sleep extends battery life, the wake up and following processes cause latency that conflicts with real time requirements [15]. Furthermore, the choice of communication protocol dictates the ``wake up tax'' of the device [16]. Traditional web protocols like HTTP are robust but carry significant header overhead, whereas lightweight protocols like MQTT are designed for exactly that missing efficiency [17]. Additionally, maintaining a consistent view of the system state in a distributed architecture is non trivial, as the CAP theorem dictates trade offs between Consistency, Partition tolerance and Availability [18]

\subsection{Research Question}
To address these challenges, this thesis poses the following primary research question:
How do MQTT and REST compare in a microservices-based architecture for real-time plant monitoring, specifically regarding latency, throughput, and data consistency, when constrained by battery-powered IoT devices [19]?

% \section{Theoretical Background}

\subsection{Microservices Architecture (MSA)}
Microservices Architecture (MSA) describes a software design approach in which an application is decomposed into small, autonomous services that communicate through lightweight protocols \cite{ref20}. Each service is responsible for a specific domain and can be deployed and scaled independently. For IoT systems, where hardware events, sensor data ingestion, user interactions and analytics occur asynchronously, this architectural style offers clear advantages in resilience and scalability \cite{ref21}.

The ``Plant Up!'' project follows this principle by organizing its backend into domain-oriented Supabase schemas: the \texttt{user\_schema} manages identities and streak logic, the \texttt{social\_media\_schema} stores posts and plant information, the \texttt{gamification} schema handles XP, quests and rewards, and the \texttt{microcontroller\_schema} stores IoT sensor readings. Each schema acts as an isolated bounded context.

A representative example is the table for environmental sensor readings:

\begin{figure}[H]
    \centering
    \includegraphics[width=0.8\textwidth]{images/Controllers table in microcontroller__schema.png}
    \caption{Controllers table in microcontroller\_schema}
    \label{fig:controllers_table}
\end{figure}

This separation prevents side effects: updating sensor data cannot interfere with social media functionality or gamification services. MSA therefore supports robustness and domain clarity.

\subsection{Core Characteristics of Microservices in IoT}
IoT systems introduce unique requirements that make certain microservice characteristics particularly important. Constrained hardware, intermittent connectivity and asynchronous event streams require an architecture that is tolerant to partial failures and scalable under variable load.

\textbf{Autonomy:}  
Each service must function independently. For example, the gamification subsystem maintains XP and level state regardless of the status of the sensor ingestion pipeline:

\begin{figure}[H]
    \centering
    \includegraphics[width=0.8\textwidth]{images/Gamification player statistics.png}
    \caption{Gamification player statistics}
    \label{fig:gamification_stats}
\end{figure}

\textbf{Loose Coupling:}  
Services communicate through clearly defined data structures. A social media post does not depend on the microcontroller service:

\begin{figure}[H]
    \centering
    \includegraphics[width=0.8\textwidth]{images/Posts table in social__media__schema.png}
    \caption{Posts table in social\_media\_schema}
    \label{fig:posts_table}
\end{figure}

\textbf{Scalability:}  
Sensor bursts occur when multiple devices wake simultaneously. Only the ingestion path needs to scale, not the entire backend.

\textbf{Resilience:}  
Since IoT devices frequently disconnect (deep sleep, Wi-Fi loss), the system must tolerate missing or delayed data. The schema boundaries allow delayed writes without blocking dependent features.

\subsection{JSON Payload Design}
JSON is used throughout Plant Up! for communication between mobile clients, backend services and IoT devices. Because the ESP32-S3 is battery-powered, payload size and structure directly influence energy consumption \cite{ref22}.

A typical payload aligned with the \texttt{Controllers} schema is:

\begin{figure}[H]
    \centering
    \includegraphics[width=0.8\textwidth]{images/Sensor payload structure.png}
    \caption{Sensor payload structure}
    \label{fig:sensor_payload}
\end{figure}

The payload reflects several design principles:
\begin{itemize}
    \item flat JSON structure to avoid deep nesting,
    \item short but descriptive field names to reduce size,
    \item consistent alignment with database schema,
    \item predictable field order for efficient parsing on constrained hardware.
\end{itemize}

Such optimizations reduce transmission time and extend device battery life.

\subsection{Internet of Things (IoT) Constraints and Hardware}
IoT hardware such as the ESP32-S3 operates under constraints that strongly influence backend architecture.

\textbf{1. Power Consumption:}  
Deep sleep drastically reduces energy usage, but each wake cycle incurs overhead due to Wi-Fi reconnection and sensor initialization. This ``wake-up tax'' conflicts with real-time monitoring needs \cite{ref23}.

\textbf{2. Intermittent Connectivity:}  
Home networks produce variable latency. Timestamped readings (\texttt{time}) allow the backend to reconstruct temporal context even with delayed uploads.

\textbf{3. Limited Compute and Memory:}  
Operations like TLS negotiation, JSON serialization and sensor polling must be minimized.

\textbf{4. Sensor Noise \& Calibration:}  
Environmental data such as soil moisture, EC and humidity fluctuate naturally. The backend must treat sensor readings as approximations and potentially smooth or validate them before using them for plant health scoring or gamification.

These constraints justify efficient protocols and eventual consistency strategies.

\subsection{Communication Protocols: MQTT vs. REST}
Communication choices fundamentally shape IoT system performance. In Plant Up!, both REST and MQTT serve different roles.

\textbf{REST (HTTPS):}  
REST is used for structured, authenticated application features such as user data, posts, quests and plant profiles. For example, plant instances are stored as:

\begin{figure}[H]
    \centering
    \includegraphics[width=0.8\textwidth]{images/Plants table.png}
    \caption{Plants table}
    \label{fig:plants_table}
\end{figure}

REST integrates seamlessly with Supabase but introduces overhead due to large HTTP headers and TLS handshakes, which is suboptimal for energy-constrained IoT devices.

\textbf{MQTT:}  
MQTT is a lightweight publish/subscribe protocol optimized for constrained hardware \cite{ref24}. Devices publish compact JSON messages to predefined topics and immediately return to deep sleep. This minimizes active Wi-Fi time and improves battery longevity.

A topic pattern suitable for Plant Up! is:

\begin{figure}[H]
    \centering
    \includegraphics[width=0.8\textwidth]{images/topic pattern for Plant Up! .png}
    \caption{Topic pattern for Plant Up!}
    \label{fig:topic_pattern}
\end{figure}

\textbf{Architectural Implication:}  
Plant Up! adopts a hybrid model:
\begin{itemize}
    \item REST for app features requiring authentication and structured queries,
    \item MQTT for energy-efficient, low-latency sensor data ingestion.
\end{itemize}

\subsection{Data Consistency and the CAP Theorem}
In distributed systems, the CAP theorem states that a system cannot simultaneously provide Consistency (C), Availability (A) and Partition Tolerance (P) \cite{ref25}. Since IoT systems must assume network partitions due to intermittent connectivity, Partition Tolerance is unavoidable.

Plant Up! prioritizes Availability and Partition Tolerance (AP), accepting that sensor readings may arrive late and the system will become consistent over time.

Upon reconnection from deep sleep, readings are inserted into the \texttt{Controllers} table:

\begin{figure}[H]
    \centering
    \includegraphics[width=0.8\textwidth]{images/Insert example for sensor readings.png}
    \caption{Insert example for sensor readings}
    \label{fig:insert_example}
\end{figure}

Analytics and gamification services operate on this eventually consistent data. This ensures:
\begin{itemize}
    \item the mobile app remains responsive,
    \item no sensor data is lost during connectivity gaps,
    \item gamification logic (e.g., XP, quests) updates when sufficient data is available.
\end{itemize}

This trade-off aligns with the practical constraints of consumer IoT devices.

% \section{Plant Up! System Architecture and Implementation}

\subsection{System Overview and Requirements}
The ``Plant Up!'' system integrates IoT hardware, microservices and a mobile application into a unified platform designed to support real-time plant monitoring and social interaction. The primary system objective is to collect environmental sensor data from consumer-grade hardware, synchronize it across distributed cloud services and present it to users in a gamified, socially engaging interface.

To achieve this, the system must satisfy three key requirements:
\begin{itemize}
    \item \textbf{Low-latency sensor synchronization:} Environmental data must be transmitted and processed fast enough to provide timely feedback on plant health.
    \item \textbf{Energy-efficient operation:} IoT devices must conserve battery life through deep sleep cycles, lightweight payloads and minimal active radio time.
    \item \textbf{Distributed consistency:} Since data is stored across multiple domain-specific schemas, the system must tolerate intermittent connectivity while ensuring eventual consistency.
\end{itemize}

These requirements shape the architectural decisions in both hardware and microservices. The system therefore follows a multi-layered design consisting of an edge node, a cloud microservices layer and a client application layer.

\subsection{Hardware Layer: The Edge Node}
The hardware layer consists of an ESP32-S3 microcontroller equipped with sensors for temperature, humidity, light intensity, soil moisture and electrical conductivity. This configuration enables comprehensive monitoring of plant health. The device operates under strict energy constraints and therefore relies heavily on deep sleep modes. Upon waking, it performs three tasks:

\begin{enumerate}
    \item Reads environmental sensors.
    \item Serializes the data into a compact JSON structure.
    \item Sends the payload to the backend via MQTT or REST before returning to deep sleep.
\end{enumerate}

The JSON payload corresponds to the structure of the \texttt{Controllers} table in the \texttt{microcontroller\_schema}:

\begin{figure}[H]
    \centering
    \includegraphics[width=0.8\textwidth]{images/Sensor payload structure.png}
    \caption{JSON payload sent by the edge node}
    \label{fig:edge_json_payload}
\end{figure}

The corresponding Supabase table is:

\begin{figure}[H]
    \centering
    \includegraphics[width=0.8\textwidth]{images/Controllers table in microcontroller__schema.png}
    \caption{Controllers table receiving IoT data}
    \label{fig:controllers_table_rx}
\end{figure}

By timestamping each measurement, the backend can reconstruct time series data even when devices experience connectivity delays, enabling eventual consistency.

\subsection{Microservice Architecture Design}
The backend adopts a microservices-inspired architecture grounded in Supabase schemas. Each schema represents a bounded context aligned with a specific domain:

\begin{itemize}
    \item \textbf{user\_schema}: Stores user profiles, streak data and virtual currency.
    \item \textbf{social\_media\_schema}: Manages posts, comments, plants and plant metadata.
    \item \textbf{microcontroller\_schema}: Stores IoT sensor measurements and device associations.
    \item \textbf{gamification}: Contains quests, user quest progress and XP statistics.
\end{itemize}

This separation allows backend services to evolve independently and prevents cross-domain interference. For example, the creation of a post does not impact sensor ingestion, and a device update does not affect quest completion logic.

An example of domain separation is evident in the \texttt{Plants} table:

\begin{figure}[H]
    \centering
    \includegraphics[width=0.8\textwidth]{images/Plants table.png}
    \caption{Plants table in social\_media\_schema}
    \label{fig:plants_table_arch}
\end{figure}

This table links user-owned plants to ideal environmental parameters in \texttt{Plant\_data}, enabling health comparisons by other services.

\subsection{Real-Time Data Synchronization Mechanism}
Real-time synchronization in ``Plant Up!'' requires balancing two competing goals: minimizing battery usage on the edge node while providing timely updates to the microservices layer. Two communication mechanisms are evaluated:

\begin{itemize}
    \item \textbf{REST (HTTPS)} offers structured, authenticated, synchronous transmission suitable for user-driven interactions but incurs significant overhead.
    \item \textbf{MQTT} provides lightweight, publish/subscribe semantics with lower transmission cost, making it more suitable for the ESP32-S3.
\end{itemize}

Sensor data is synchronized using a hybrid approach. IoT controllers publish sensor payloads through MQTT for efficiency. The backend processes the incoming data and stores it in \texttt{microcontroller\_schema.Controllers}. User-facing services such as the mobile app retrieve the aggregated data via REST.

A typical ingestion operation is represented by:

\begin{figure}[H]
    \centering
    \includegraphics[width=0.8\textwidth]{images/Insert example for sensor readings.png}
    \caption{Insert operation for real-time sensor synchronization}
    \label{fig:insert_op_sync}
\end{figure}

Since devices may reconnect sporadically, synchronization follows an eventually consistent model rather than strict ordering guarantees.

\subsection{Social Gamification Logic Implementation}
Gamification is integrated into the Plant Up! experience to motivate sustained engagement. The gamification subsystem relies on three core tables:

\begin{itemize}
    \item \textbf{Quests}: Defines daily and weekly goals.
    \item \textbf{User\_quests}: Tracks per-user quest progression.
    \item \textbf{player\_stats}: Stores XP and level totals.
\end{itemize}

For example, the quests table is defined as:

\begin{figure}[H]
    \centering
    \includegraphics[width=0.8\textwidth]{images/Quests table in gamification schema.png}
    \caption{Quests table in gamification schema}
    \label{fig:quests_table}
\end{figure}

When a user completes an action, such as watering a plant or posting an update, the backend increments their quest progress:

\begin{figure}[H]
    \centering
    \includegraphics[width=0.8\textwidth]{images/User quest progression.png}
    \caption{User quest progression}
    \label{fig:user_quest_progression}
\end{figure}

Once the target count is met, XP is awarded:

\begin{figure}[H]
    \centering
    \includegraphics[width=0.8\textwidth]{images/Awarding XP to player_stats.png}
    \caption{Awarding XP to player\_stats}
    \label{fig:xp_award}
\end{figure}

This modular design ensures that the gamification logic remains independent of the sensor ingestion pipeline, social media features and plant data system.

% \section{Experimental Evaluation of MQTT vs. REST}

\subsection{Introduction to the Experiment}
To provide a rigorous answer to the research question—How do MQTT and REST compare in a microservices-based architecture for real-time plant monitoring?—this thesis employs an empirical experimental approach. The ``Plant Up!'' system, as detailed in Chapter 3, serves as the testbed. The evaluation focuses on quantifying the three critical performance vectors defined in the problem statement: Latency, Throughput, and Energy Efficiency.

This chapter details the experimental setup, the measurement methodology, and the specific scenarios designed to stress-test the protocols under conditions mimicking a real-world urban gardening environment (e.g., unstable Wi-Fi, battery constraints).

\subsection{Experimental Setup}
The testbed is constructed to isolate the communication protocol as the single independent variable. Both the hardware (Edge) and the backend (Cloud) remain constant, with only the application layer transport mechanism toggling between MQTT (v3.1.1) and REST (HTTP/1.1).

\subsubsection{Hardware Configuration (Edge Layer)}
The experiments utilize the ESP32-S3-DevKitC-1, selected for its relevance to the ``Plant Up!'' production specification.
\begin{itemize}
    \item \textbf{Microcontroller}: ESP32-S3 (Xtensa\textregistered\ 32-bit LX7 dual-core, 240 MHz).
    \item \textbf{Network Interface}: Integrated 2.4 GHz Wi-Fi (802.11 b/g/n).
    \item \textbf{Power Measurement}: Nordic Semiconductor Power Profiler Kit II (PPK2), set to ``Source Meter'' mode with a sampling rate of 100ksps (kilo-samples per second) to capture transient current spikes during Wi-Fi transmission.
    \item \textbf{Sensors}: Simulated sensor data is used during load testing to ensure deterministic payload sizes, eliminating variance caused by sensor read times.
\end{itemize}

\subsubsection{Backend Environment (Cloud Layer)}
The microservices backend is hosted on a local Kubernetes cluster (Minikube) to eliminate internet service provider (ISP) jitter from the latency measurements.
\begin{itemize}
    \item \textbf{Broker}: Eclipse Mosquitto (v2.0.11) deployed as a Docker container.
    \item \textbf{API Gateway}: NGINX (v1.21) acting as the reverse proxy for REST requests.
    \item \textbf{Database}: Supabase (v2.0) for time-series storage.
    \item \textbf{Network Emulation}: \texttt{tc} (Traffic Control) is used on the Linux host to simulate packet loss (2-5\%) and added latency (50-100ms), replicating poor residential Wi-Fi signal strength (-80dBm).
\end{itemize}

The following infrastructure configuration ensures both protocols are available simultaneously for A/B testing:
% Image removed


\subsection{Methodology and Metrics}

\subsubsection{Metric 1: End-to-End Latency ($L_{e2e}$)}
Latency is defined as the time elapsed between the generation of a sensor reading at the Edge Node ($t_{gen}$) and its persistence in the Supabase database ($t_{ack}$).
\[ L_{e2e} = t_{ack} - t_{gen} \]
To measure this accurately without relying on un-synchronized clocks (clock drift between ESP32 and Server), we utilize a ``Round Trip Time'' (RTT) approach for the benchmark. The ESP32 sends a message and waits for an application-layer acknowledgement from the server.
\begin{itemize}
    \item \textbf{MQTT}: Time from \texttt{publish()} to arrival of \texttt{PUBACK} (QoS 1).
    \item \textbf{REST}: Time from \texttt{http.POST()} to return of HTTP 200 OK.
\end{itemize}
Note: In the REST implementation, \texttt{http.begin()} often initiates the TCP handshake, heavily penalizing the latency score if Keep-Alive is not active.

\begin{center}
    \includegraphics[width=0.8\textwidth]{images/MQTT RTT Measurement (PUBACK timing).png}
    \captionof{figure}{MQTT RTT Measurement (PUBACK timing)}
    \label{fig:temp_image_rtt}
\end{center}
\subsubsection{Metric 2: Throughput and Congestion}
Throughput is evaluated by increasing the message frequency ($f_{msg}$) from 1 Hz to 100 Hz. The metric is Successful Messages Per Second (SMPS).
This test simulates a ``Broadcast Event'' where the ``Social Service'' might request immediate status updates from all plants in a Guild (e.g., during a ``Watering Party'' game event). We observe the point at which packet loss exceeds 1\% or the ESP32's queue overflows.

\subsubsection{Metric 3: Energy Consumption}
Using the DEBUG\_PIN triggers the Logic Analyzer to capture the exact current draw during the transmission window.
\begin{center}
    \includegraphics[width=0.8\textwidth]{images/GPIO Toggle for Measurement Trigger.png}
    \captionof{figure}{GPIO Toggle for Measurement Trigger}
    \label{fig:temp_image_gpio}
\end{center}

To evaluate Data Consistency, we induce network partitions (simulating a user walking out of Wi-Fi range with the portable sensor unit).

\textbf{Scenario}: The device generates 100 messages while disconnected.
\begin{itemize}
    \item \textbf{REST Behavior}: The \texttt{HTTPClient} returns connection errors. We measure how many messages are lost vs. how many are successfully buffered in the ESP32's limited RAM (Ring Buffer) and queued for reconnection.
    \item \textbf{MQTT Behavior}: We test QoS 1 and QoS 2 with Persistent Sessions (CleanSession=false). The Broker should queue messages destined for the subscriber, but the Edge Node must also queue messages destined for the Cloud.
\end{itemize}

This evaluation specifically looks at the implementation of the Outbox Pattern on the embedded device, which is crucial for the ``Eventual Consistency'' required by the CAP theorem analysis in Chapter 2.4.

\begin{center}
    \includegraphics[width=0.8\textwidth]{images/Outbox Pattern Buffering on ESP32.png}
    \captionof{figure}{Outbox Pattern Buffering on ESP32}
    \label{fig:temp_image_outbox}
\end{center}


\section{Introduction}

\subsection{Background}

\subsubsection{Microservices in Gardening}
\begin{itemize}
    \item Paradigm shift, individual plant care, community interaction, shared data, digital transformation, nature meets tech, social platforms, sensor integration, remote monitoring, enthusiast community, collaborative gardening, data-driven plant care, IoT connectivity
    \item \textbf{[INSERT FIGURE: Concept Art of Social Gardening with IoT]}
\end{itemize}

\subsubsection{The Latency Challenge: Real-time Notification vs. Cloud Round-Trips}
\begin{itemize}
    \item Technical hurdles, delay management, sensor reading, user notification delay, round-trip time (RTT), network jitter, cloud infrastructure, internet dependency, timely intervention, timing criticality, processing lag, signal propagation, feedback loops
    \item \textbf{[INSERT DIAGRAM: Round-Trip Time Visualization (Sensor $\to$ Cloud $\to$ Push Notification)]}
\end{itemize}

\subsubsection{Dependency Risks: What happens to the plant when the Microservice is down?}
\begin{itemize}
    \item Centralized architecture, single point of failure, microservice downtime, plant health risks, missed alerts, service outages, server crashes, connectivity loss, maintenance windows, failover mechanisms, contingency planning, disaster recovery
\end{itemize}

\subsection{Research Question}

\subsubsection{Primary Question}
\begin{itemize}
    \item \textbf{"What are the trade-offs between edge and cloud processing for real-time plant monitoring in consumer gardening systems?"}
    \item Edge computing benefits, Cloud computing benefits, latency vs. throughput, cost vs. complexity, data integrity, system reliability, user experience impact
\end{itemize}

\subsection{Scope and Limitations}

\subsubsection{Focus on Consumer Hardware (ESP32/ESP8266)}
\begin{itemize}
    \item Consumer-grade electronics, ESP32 capabilities, ESP8266 limitations, cost-effectiveness, accessibility, hobbyist market, maker community, hardware constraints, memory limits, processing power, WiFi modules, microcontroller adoption
    \item \textbf{[INSERT PICTURE: ESP32/ESP8266 Microcontroller Setup]}
\end{itemize}

\subsubsection{Comparison limited to Processing Location (Edge vs. Cloud), not Transport Protocols (MQTT assumed)}
\begin{itemize}
    \item Scope boundary, processing location focus, edge computing, cloud computing, transport layer abstraction, MQTT protocol, standard compliance, exclusion of CoAP/HTTP, architectural comparison, logic placement, fixed communication variables
   
\end{itemize}

\section{Theoretical Framework}

\subsection{Modern IoT Backends}

\subsubsection{Backend-as-a-Service (BaaS): The Role of Supabase in IoT}
\begin{itemize}
    \item Backend-as-a-Service, BaaS, Supabase architecture, real-time subscriptions, websockets, rapid development, managed infrastructure, scalability, postgres-native, authentication, database APIs, serverless paradigm, unified backend
    \item \textbf{[INSERT DIAGRAM: Supabase Realtime Architecture Flow]}
    \item Source: \url{https://metadesignsolutions.com/supabase-vs-firebase-when-to-choose-open-source-postgres-for-your-apps-scalability/#:~:text=Supabase%20is%20an%20open%2Dsource,power%20of%20a%20relational%20database.}
\end{itemize}

\subsubsection{Time-Series Databases: Why standard SQL fails for sensor streams}
\begin{itemize}
    \item Relational database limitations, high-velocity ingestion, storage bloat, index inefficiency, B-tree performance, standard SQL constraints, sensor data volume, continuous streams, data retention issues, query degradation
    \item Source: \url{https://milvus.io/ai-quick-reference/what-are-the-limitations-of-relational-databases}
\end{itemize}

\subsubsection{The TimescaleDB Extension: Hypertables and Chunking explained}
\begin{itemize}
    \item PostgreSQL extension, TimescaleDB, hypertables, automatic partitioning, data chunking, temporal locality, high-performance inserts, time-series optimization, compression algorithms, continuous aggregates, SQL compatibility, retention policies
    \item \textbf{[INSERT DIAGRAM: Hypertable vs. Normal Table Storage]}
    \item Source: \url{https://www.tigerdata.com/learn/the-best-time-series-databases-compared}
\end{itemize}

\subsubsection{Object Storage and Media Handling in Microservices}
\begin{itemize}
    \item Unstructured data storage, Supabase Storage, S3 compatibility, scalable object storage, media asset management, separation of concerns (DB vs. Storage), CDN caching strategies, secure file access policies, handling large binary files (blobs)
    \item Source: \url{https://github.com/supabase/storage}
\end{itemize}

\subsection{Microservices in IoT}

\subsubsection{The "Gateway Pattern": Using a C\# Service to bridge MQTT and Database}
\begin{itemize}
    \item Gateway design pattern, architectural patterns, C\# middleware, .NET Core, MQTT broker, database bridge, decoupling services, protocol translation, message buffering, security boundary, data normalization, traffic throttling
    \item \textbf{[CODE SNIPPET: C\# Worker Service Basic Structure]}
    \item Source: \url{https://microservices.io/patterns/apigateway.html}
\end{itemize}

\subsubsection{Service Inter-communication: Synchronous vs. Asynchronous patterns}
\begin{itemize}
    \item Inter-service communication, synchronous REST, asynchronous messaging, message queues, event-driven architecture, request-response, immediate consistency, eventual consistency, system coupling, latency impact, blocking operations
    \item Source: \url{https://www.geeksforgeeks.org/system-design/microservices-communication-patterns/}
\end{itemize}

\subsection{Edge Computing Patterns}

\subsubsection{Cloud Computing, Fog Computing, and Edge Computing}
\begin{itemize}
    \item \textbf{Cloud Computing:} Centralized processing, high computational power, high latency, infinite storage, global access.
    \item \textbf{Fog Computing:} Intermediate layer, gateway processing, local area network, bridging Edge and Cloud, reduced latency compared to Cloud.
    \item \textbf{Edge Computing:} Source processing, on-device logic, microcontroller level, lowest latency, real-time response, limited resources.
    \item \textbf{[INSERT FIGURE: Cloud vs. Fog vs. Edge Architecture Pyramid]}
\end{itemize}

\subsubsection{"Thick Edge" vs. "Thin Edge": How much logic should the microcontroller hold?}
\begin{itemize}
    \item Edge architecture, Thick Edge, Thin Edge, logic distribution, processing offload, local autonomy, bandwidth conservation, resource constraints, computational power, firmware complexity, distributed intelligence, architectural trade-offs, remote management
\end{itemize}

\section{The PlantUp Architecture}

\subsection{System Overview}

\subsubsection{High-Level Diagram: From Soil Sensor to React Native App}
\begin{itemize}
    \item Full-stack architecture, system topology, soil moisture sensors, analog-to-digital conversion, capacitive sensing, React Native framework, mobile application, user interface, cloud synchronization, end-to-end data flow, interactive dashboard
    \item \textbf{[INSERT FIGURE: High-Level System Architecture Diagram]}
    \item Source:  \url{https://www.geeksforgeeks.org/postgresql/postgresql-schema/} \newline \url{https://www.geeksforgeeks.org/system-design/architecture-of-a-system/}
\end{itemize}

\subsection{The Microservices Layer (Cloud)}

\subsubsection{User \& Social Media Services: Gamification and Content Sharing}
\begin{itemize}
    \item Social microservices, gamification engine, PostgreSQL schemas, user profiles, experience points (XP), leaderboards, achievement system, social graphs, user interactions, relational data modeling, reward logic, competition mechanics
    \item \textbf{Social Media Features:} Video upload pipeline, automatic thumbnail generation, post creation, Supabase Storage buckets, media metadata, content delivery network (CDN) integration, secure signed URLs, social feed algorithms
    \item \textbf{[INSERT DIAGRAM: Gamification Database Schema (ERD)]}
    \item \textbf{[INSERT DIAGRAM: Video Upload and Thumbnail Generation Flow]}
    \item Source: \url{https://www.cloudflare.com/learning/cdn/what-is-a-cdn/} \newline \url{https://supabase.com/docs/guides/storage/cdn/fundamentals}
\end{itemize}

\subsubsection{Webshop Service: Shopify Integration}
\begin{itemize}
    \item Shopify platform, SaaS e-commerce, API integration, inventory management, external service, seamless user experience, checkout process, webhook sync, product catalog, managed hosting, third-party plugin ecosystem
    \item \textbf{[INSERT SCREENSHOT: PlantUp Shop Concept on Shopify]}
    \item Source: \url{https://www.cloudflare.com/de-de/learning/cloud/what-is-saas/} \newline \url{https://www.techtarget.com/searchcloudcomputing/definition/Software-as-a-Service}
\end{itemize}

\subsubsection{Microcontroller Management Service: The C\# .NET Core Worker}
\begin{itemize}
    \item Background worker, .NET ecosystem, device management, command orchestration, state synchronization, keep-alive monitoring, message processing, centralized control, device registry, firmware updates (OTA), scalable architecture
    \item \textbf{[CODE SNIPPET: C\# Device State Manager Class]}
    \item Source: \url{https://learn.microsoft.com/en-us/aspnet/core/fundamentals/host/hosted-services?view=aspnetcore-10.0&tabs=visual-studio}
\end{itemize}

\subsection{The Data Persistence Layer (Supabase)}

\subsubsection{Unified Database Strategy: Single Postgres instance with logical schema separation}
\begin{itemize}
    \item Database consolidation, single instance, logical isolation, schema-based multitenancy, operational simplicity, reduced overhead, unified backup, cross-schema querying, management efficiency, Postgres roles, access control
 
\end{itemize}

\subsubsection{The microcontroller\_data Schema: Utilizing the TimescaleDB Extension}
\begin{itemize}
    \item Dedicated schema, TimescaleDB integration, sensor data storage, time-series optimization, optimized querying, high-volume tables, data partitioning, specialized indexing, analytical queries, storage efficiency, historical data
    \item \textbf{[CODE SNIPPET: SQL Creation Script for TimescaleDB Hypertable]}
    \item Source: \url{https://www.influxdata.com/time-series-database/} \newline \url{https://github.com/timescale/timescaledb}
\end{itemize}

\subsubsection{Hypertable Design: Partitioning strategy for millions of sensor readings}
\begin{itemize}
    \item Hypertable architecture, data partitioning, time intervals, chunk management, scalable storage, query performance, massive datasets, millions of rows, retention management, automated maintenance, disk I/O optimization
    \item \textbf{[INSERT DIAGRAM: Visualizing Hypertable Chunks based on Time]}
    \item Source: \url{https://www.cloudthat.com/resources/blog/scaling-time-series-data-with-timescaledb-hypertables} 
\end{itemize}

\section{Methodology}

\subsection{Experimental Architectures}

\subsubsection{Architecture A: Cloud-Centric (The "Connected" Plant)}
\begin{itemize}
    \item Cloud-centric design, connected model, centralized decision making, MQTT transmission, C\# service logic, historical querying, time-weighted averages, notification latency, server-side processing, dumb terminal, remote monitoring, bandwidth dependence
    \item \textbf{[INSERT DIAGRAM: Architecture A Data Flow (Sensor $\to$ Cloud $\to$ Push Notification)]}
    \item Source: \url{https://www.ibtimes.com/architecting-connected-retail-future-cloud-centric-data-fabric-omnichannel-agility-3793476}
\end{itemize}

\subsubsection{Architecture B: Edge-Centric (The "Smart" Plant)}
\begin{itemize}
    \item Edge-centric design, smart device, distributed intelligence, local processing, independent operation, C++ logic, asynchronous logging, eventual consistency, disconnect tolerance, local status indication, autonomous state monitoring, bandwidth efficiency
    \item \textbf{[INSERT DIAGRAM: Architecture B Data Flow (Sensor $\to$ Edge $\to$ Local Indication/Async Log)]}
    \item Source: \url{https://www.mdpi.com/1424-8220/20/3/892}
\end{itemize}

\subsection{Implementation Details}

\subsubsection{Edge Firmware (C++): Sensor Calibration and Power Management}
\begin{itemize}
    \item Firmware development, C++ programming, embedded systems, capacitive moisture sensing, signal smoothing, debouncing, deep sleep modes, battery optimization, data packetization, interrupt handling, robust error recovery
    \item \textbf{[CODE SNIPPET: C++ ESP32 Deep Sleep \& Sensor Reading]}
\end{itemize}

\subsubsection{Cloud Service (C\#): Implementation of the Supabase.Client and Supabase.Realtime listeners}
\begin{itemize}
    \item Cloud service implementation, C\# coding, Supabase SDK, real-time listeners, websocket subscription, event handling, state synchronization, database triggers, remote procedure calls, responsive backend, client library integration
    \item \textbf{[CODE SNIPPET: C\# Supabase Realtime Listener Setup]}
    \item Source: \url{https://github.com/supabase/realtime}
\end{itemize}

\subsubsection{Database Schema: Definition of sensor\_readings hypertable}
\begin{itemize}
    \item Schema definition, DDL statements, hypertable creation, column types, indexing strategy, partition keys, compression settings, data persistence, storage optimization, analytical capabilities, query planning
    \item \textbf{[CODE SNIPPET: SQL for Hypertable Creation and Compression Policy]}
    \item Source: \url{https://gunesramazan.medium.com/managing-iot-heartbeat-data-with-postgresql-timescaledb-c4984b17a341}
\end{itemize}

\subsubsection{Power Supply Design and Consumption Analysis}
\begin{itemize}
    \item \textbf{Worst Case Scenario:} Continuous operation, all sensors active, WiFi transmission 24/7, ~168 mA continuous draw, 4032 mAh daily consumption, theoretical maximum, stress testing benchmarks
    \item \textbf{Realistic Strategy (Deep Sleep):} Cyclic operation, 10-minute reporting interval, ~2-5 mA average consumption, user-configurable frequency, deep sleep (~0.1 mA), active burst (~150 mA for 3s), efficient power management
    \item \textbf{Power Architecture:} Rechargeable LiPo battery (3.7V), Dual charging path (USB-C + Solar), BQ24074 Power Management Module, automatic source switching, stable voltage regulation, outdoor autonomy
    \item \textbf{Component Specifications:} LiPo Capacity (3000-5000 mAh), Solar Panel (5-6V, 1-2W, outdoor-proof), Voltage Regulator (Integrated)
    \item \textbf{Battery Life Projections:} 
        \begin{itemize}
            \item 3000 mAh Battery: ~3 mA avg $\rightarrow$ ~1000 hours $\rightarrow$ approx. 41 days
            \item 5000 mAh Battery: ~3 mA avg $\rightarrow$ ~1666 hours $\rightarrow$ approx. 69 days (>2 months)
        \end{itemize}
    \item \textbf{[INSERT GRAPH: Power Consumption Profile (Deep Sleep vs Active Burst)]}
    \item Source: \url{https://running-system.atlassian.net/wiki/spaces/RS/pages/126550051/Power+Management+Research}
\end{itemize}

\subsection{Test Scenarios}
\begin{itemize}
    \item Source: \url{https://www.geeksforgeeks.org/software-testing/system-testing/}
    \newline
    \url{https://github.com/timescale/tsbs}
    \newline
    \url{https://www.bytebase.com/blog/postgres-row-level-security-limitations-and-alternatives/}
    \newline
    \url{https://supabase.com/docs/guides/realtime/benchmarks}
\end{itemize}



\subsubsection{Scenario 1: Ideal Conditions}
\begin{itemize}
    \item Baseline testing, ideal environment, low latency, zero packet loss, stable connectivity, theoretical maximums, benchmarking, optimal performance, clean signal, control group, reference measurements
\end{itemize}

\subsubsection{Scenario 2: Database Load (Simulating 1,000 concurrent writes to TimescaleDB)}
\begin{itemize}
    \item Load testing, stress test, concurrent users, high ingestion rate, database bottlenecks, resource contention, write blocking, 1000 connections, performance degradation, scalability limits, system stability, latency spike
    \item \textbf{[INSERT PICTURE: Load Testing Setup / Console Logs]}
\end{itemize}

\subsubsection{Scenario 3: Service Outage (Simulating a crash of the C\# Microservice)}
\begin{itemize}
    \item Resilience testing, fault injection, service crash, chaos engineering, availability test, outage simulation, watchdog timers, system recovery, safety verification, data loss analysis, fail-safe behavior
\end{itemize}

\section{Results and Data Analysis}

\subsection{Latency Analysis}

\subsubsection{Edge Response Time (<50ms) vs. Cloud Round Trip (>200ms + DB Query Time)}
\begin{itemize}
    \item Quantitative analysis, edge speed, sub-50ms response, cloud delay, >200ms latency, network overhead, database query time, round-trip metrics, comparison charts, performance penalty, real-time constraints, user experience impact
    \item \textbf{[INSERT GRAPH: Comparison Bar Chart of Alert Generation Time (Edge vs. Cloud)]}
    \item Source: \url{https://ifactoryapp.com/blog/real-time-ai-cloud-latency-manufacturing}
\end{itemize}

\subsection{Database Performance}

\subsubsection{Impact of High Ingestion Rates on C\# Service Query Performance}
\begin{itemize}
    \item Performance impact, ingestion throughput, query latency, lock contention, C\# service load, resource usage, CPU/Memory metrics, read-write conflict, optimization analysis, database tuning, concurrency issues
    \item \textbf{[INSERT GRAPH: Query Latency vs. Ingestion Rate]}
\end{itemize}

\subsubsection{Storage Efficiency: Raw Data (Cloud Decision) vs. Batched Aggregates (Edge Decision)}
\begin{itemize}
    \item Storagemetrics, raw data volume, aggregation efficiency, payload size, storage costs, bandwidth consumption, cloud storage, batched transmission, data compression, economy of scale, archival strategy
    \item \textbf{[INSERT CHART: Projected Storage Costs Over Time]}
    \item Source: \url{https://metadeskglobal.com/from-raw-sensor-data-to-intelligent-automation/}
    \newline \url{https://stonefly.com/blog/iot-data-storage-architectures-databases/}
\end{itemize}

\subsection{Reliability and Resilience}

\subsubsection{Alert Delivery Success Rates during a simulated 1-hour internet outage}
\begin{itemize}
    \item Reliability metrics, delivery success analysis, outage simulation, local data buffering, resync capability, data integrity, critical failure, connectivity loss, edge resilience, robust design, message queuing
    \item \textbf{[INSERT TIMELINE: Data Sync Recovery after 1-Hour Outage]}
    \item  Source: \url{https://www.opal-rt.com/blog/5-essential-computer-simulation-techniques-for-systems-engineers/}
\end{itemize}

\subsection{Gamification Synchronization}

\subsubsection{The ``Lag'' between Action (Watering) and Reward (XP Update in Social Service)}
\begin{itemize}
    \item User experience, feedback delay, action-reward loop, gamification lag, sync latency, psychological impact, interface responsiveness, XP updates, social satisfaction, system consistency, near real-time
    \item \textbf{[INSERT DIAGRAM: Sequence Diagram of Action-to-Reward Delay]}
    \item  Source: \url{https://supabase.com/docs/guides/functions}
\end{itemize}

\section{Discussion}

\subsection{The "Hybrid" Sweet Spot for PlantUp}

\subsubsection{Using Edge for Real-time Monitoring and Cloud for Engagement (XP/Leaderboards)}
\begin{itemize}
    \item Architectural synthesis, hybrid model, best of both worlds, edge reliability, cloud scalability, critical vs. non-critical, monitoring logic, engagement layer, strategic separation, optimal design, computing paradigm balance
    \item \textbf{[INSERT DIAGRAM: The Hybrid "Sweet Spot" Logic Distribution]}
    \item Source: \url{https://ieeexplore.ieee.org/document/11158770}
\end{itemize}

\subsection{Database Considerations}

\subsubsection{Benefits of Supabase's unified access for the React Native App}
\begin{itemize}
    \item Developer experience, unified API, frontend simplification, React Native integration, simplified auth, reduced complexity, rapid prototyping, coherent data model, streamlined access, architectural elegance
    \item Source: \url{https://taglineinfotech.com/blog/which-backend-is-best-for-react-native/}
\end{itemize}

\subsubsection{Trade-offs of querying TimescaleDB for real-time control loops}
\begin{itemize}
    \item Design trade-offs, real-time query risks, database dependency, latency variability, connection overhead, system stability, control loop jitter, architectural decisions, load implications, performance vs. consistency
    \item Source: \url{https://www.tigerdata.com/blog/timescaledb-vs-influxdb-for-time-series-data-timescale-influx-sql-nosql-36489299877}
\end{itemize}

\subsection{Complexity vs. Cost}

\subsubsection{Development cost of maintaining logic in two places (C++ and C\#) vs. centralized C\# logic}
\begin{itemize}
    \item Cost-benefit analysis, maintenance overhead, code duplication, technological diversity, C++ vs C\#, staffing requirements, development velocity, operational complexity, total cost of ownership, robustness justification, long-term viability
    \item \textbf{[INSERT TABLE: Cost/Complexity Comparison Matrix]}
    \item Source: Chapter 2.1 \url{https://flame-challenge.authorea.com/users/982330/articles/1348311/master/file/data/wileyNJDv5_AMA/wileyNJDv5_AMA.pdf}
    \newline  \url{https://www.baeldung.com/cs/distributed-systems-thin-vs-thick-clients}
    \newline \url{https://www.einfochips.com/blog/understanding-risks-in-over-the-air-firmware-upgrade-for-automotives-including-evs/}
\end{itemize}

\section{Conclusion}

\subsection{Summary of Findings}
\begin{itemize}
    \item Research summary, key takeaways, latency findings, reliability confirmation, architectural verdict, performance benchmarks, hypothesis validation, data-driven conclusions, system evaluation, project goal achievement
\end{itemize}

\subsection{Architectural Recommendations for PlantUp}

\subsubsection{Adopt "Edge-First" Monitoring to guarantee data integrity}
\begin{itemize}
    \item Strategic recommendation, Edge-First philosophy, integrity priority, decentralized monitoring, robustness first, fail-safe architecture, local buffering, connectivity independence, critical path optimization
\end{itemize}

\subsubsection{Use Supabase Edge Functions (optional) vs. C\# Service for lighter tasks}
\begin{itemize}
    \item Alternative architecture, serverless functions, Supabase Edge Functions, lightweight logic, cost reduction, operational simplicity, scaling flexibility, C\# service replacement, modernization, reduced maintenance
\end{itemize}

\subsection{Future Work (e.g., Predictive Health Alerts using Machine Learning on TimescaleDB data)}
\begin{itemize}
    \item Future directions, predictive analytics, machine learning, AI integration, smart health alerts, user guidance, data mining, TimescaleDB analytics, proactive care, intelligent systems, next-generation IoT
    \item \textbf{[INSERT CONCEPT ART: Future AI Dashboard]}
\end{itemize}

\subsection{Connection to the Next Chapter}

Since the entire communication architecture is based on MQTT to ensure real-time performance and reliability, the system avoids the complexities of hybrid protocol management. The following chapter details the trade-offs between MQTT and REST, and how the IoT Sync mechanism leverages this MQTT foundation to ensure seamless data synchronization.

%!TEX root = ../Diplomarbeit.tex
\chapter[Combining Nature and Technology]{Combining Nature and Technology: How Sensor-Based and Data-Driven Technologies Enhance Plant Monitoring and Care}
\pagenumbering{arabic}
\setcounter{page}{1}

\section{Introduction (approx. 1 Page)}

\subsection{Motivation and Objectives}

Houseplants are a common feature in modern households, yet maintaining them over longer periods remains challenging for many people. Gardening reports and market surveys indicate that a large share of indoor plants does not survive its first year, despite regular care attempts \cite{terrariumTribeStats2024}. One of the most common reasons is improper care rather than neglect. In particular, horticultural experts identify overwatering as the leading cause of houseplant failure, frequently resulting in yellowing leaves, wilting, and root rot \cite{lsuWaterWisely}. This indicates a fundamental issue: plants often suffer because their actual needs are misunderstood.

A key reason for this misunderstanding lies in the limitations of human human judgment. Soil moisture is commonly assessed by touching the surface of the soil, even though moisture levels can differ significantly between the upper layer and the root zone \cite{lsuWaterWisely}. As a result, soil may appear dry while deeper layers remain saturated, which increases the risk of root diseases and root rot \cite{rootRotsWisc}. These processes occur below the surface and are therefore difficult to detect without technical support.

Human perception of light conditions is similarly unreliable. The visual system adapts to changing brightness levels (light adaptation), which often leads to an overestimation of the usable light available in indoor environments \cite{frontiersLightAdaptation}. In plant science, usable light is described as photosynthetically active radiation (PAR), referring to the wavelength range between 400 and 700 nanometers that plants can use for photosynthesis \cite{nasaPAR}. A room may appear bright to humans while still providing insufficient light for healthy plant growth.

At the same time, sensor technology and Internet of Things (IoT) applications are becoming increasingly widespread. The global smart home market is experiencing strong growth, driven mainly by automation in areas such as lighting, climate control, and energy management \cite{marketsandmarketsSmartHome2024,precedenceSmartHome2025}. Despite this development, plant care in private households remains largely intuition-based. This gap motivates the development of \textit{Plant Up!}, a low-cost IoT system that translates environmental sensor data into clear and actionable care recommendations.

The motivation for this project is also connected to sustainability. Agriculture accounts for approximately 70\,\% of global freshwater withdrawals, highlighting the importance of efficient water use \cite{faoWaterSustainableFood,worldbankFreshwater70}. Although \textit{Plant Up!} targets home gardening rather than large-scale agriculture, the same principle applies: data-driven monitoring supports more precise watering decisions and reduces unnecessary resource consumption.

The primary objective of this thesis is to show that effective plant monitoring does not require expensive or specialized equipment. Using affordable, consumer-grade hardware such as capacitive soil moisture sensors, digital environmental sensors, and a microcontroller-based data acquisition system, this work demonstrates how common care mistakes can be reduced. In doing so, it aims to improve plant survival and provide users with a clearer understanding of the environmental conditions that influence plant health.
\subsection{Research Question and Relevance}

Based on the challenges outlined above, this thesis addresses the following research question:

\textit{How can a low-cost, sensor-based system be designed to make plant monitoring more objective and user-friendly?}

This question focuses on practical application rather than theoretical considerations. The goal is not to replace horticultural expertise, but to support everyday plant owners by providing objective information that simplifies decision-making. By translating sensor measurements into clear feedback, the system helps users recognize when action is necessary and when intervention would be counterproductive.

The relevance of this research can be viewed from several perspectives. From an environmental perspective, this project helps save water. Many people water their plants based on routine or guessing, which often leads to waste. By using sensors, water is only used when the plant actually needs it, supporting a more sustainable approach \cite{faoWaterSustainableFood,worldbankFreshwater70}.

From a practical perspective, many plant owners are frustrated when their plants die despite regular care. By visualizing environmental conditions and offering clear care recommendations, successful plant care becomes easier, as users do not need detailed botanical knowledge.

The topic is also relevant from a technological perspective. Advances in microcontrollers, sensor accuracy, and wireless communication make it possible to implement functional monitoring systems using affordable consumer-level hardware. This thesis demonstrates how such technologies can be combined into an accessible Internet of Things (IoT) solution for everyday plant care.

To understand why these measurements are essential, the following chapter examines the environmental factors that directly influence plant growth and health.


\section{Environmental Factors in Plant Growth}

% Bedeutung von Bodenfeuchtigkeit, Temperatur, Licht und Luftfeuchtigkeit
\subsection{Importance of Soil Moisture, Temperature, Light, and Humidity}

% TODO: Soil moisture: Important for nutrient absorption, risk of root rot with waterlogging
% TODO: Light: Photosynthesis driver (difference between shade vs. sun plants)
% TODO: Temperature/Humidity: Plant transpiration (evaporation)


% Auswirkungen von Umweltfaktoren auf Pflanzenwachstum
\subsection{Effects of Environmental Factors on Plant Growth}

% TODO: What happens with deficiency? (Wilting, growth stop)
% TODO: What happens with excess? (Mold, burns)


% Notwendigkeit präziser Messungen
\subsection{Necessity of Precise Measurements}

% TODO: Why is not "feeling with your finger" always sufficient? (Subjective perception vs. objective data)


\section{Sensor Technologies for Plant Monitoring}

% Überblick über gängige Sensortypen
\subsection{Overview of Common Sensor Types}

% TODO: Soil moisture: Compare resistive (rust quickly) vs. capacitive sensors (better for long-term use)
% TODO: Temperature/Humidity: Combo sensors (e.g., DHT11 vs. DHT22 vs. BME280)
% TODO: Light: Photoresistor (LDR) vs. digital light sensors (BH1750)


% Grenzen und typische Fehlerquellen
\subsection{Limitations and Typical Error Sources}

% TODO: Influence of different substrates and soil types on measurements
% TODO: Wear, corrosion, and contamination of sensor surfaces
% TODO: Measurement fluctuations due to temperature changes and electrical interference
% TODO: Necessity to critically evaluate sensor data and compare with reality (plant condition)


% Auswahlkriterien
\subsection{Selection Criteria}

% TODO: Create decision matrix: Why did you choose exactly these sensors for "Plant Up!"?
% TODO: Reasons: Price-performance ratio, corrosion resistance, availability for Arduino/ESP


\section{Data Acquisition and Processing}

% Messdatenerfassung mit Mikrocontrollern
\subsection{Data Acquisition with Microcontrollers}

% TODO: Introduction to the controller (e.g., ESP32 or Arduino). Why this one? (e.g., ESP32 because of integrated Wi-Fi/Bluetooth)
% TODO: Wiring: How do analog/digital signals enter the chip (ADC, I2C, GPIO)

% Signalverarbeitung und Datenübertragung
\subsection{Signal Processing and Data Transmission}

% TODO: Problem "Noise": Mention that raw data often fluctuates and how you smooth it (e.g., average of 10 measurements)
% TODO: Transmission path: Does the controller send data via Wi-Fi (MQTT, HTTP request) or Bluetooth?


% Speaker in Datenbanken oder Cloud-Systemen
\subsection{Storage in Databases or Cloud Systems}

% TODO: Where does the data end up? (Local SD card, MySQL database, InfluxDB, or IoT platform like ThingSpeak)


% Grundlagen der Datenbereinigung
\subsection{Fundamentals of Data Cleaning}

% TODO: Handling faulty measurements (e.g., sensor unplugged = value 0, should not distort the average)


\section{Data-Driven Decision Making in Plant Care}

% Nutzung gesammelter Daten zur Pflegeoptimierung
\subsection{Using Collected Data for Care Optimization}

% TODO: Definition of thresholds. Example: "If soil moisture < 30%, then alarm"


% Ableitung von Pflegeempfehlungen
\subsection{Deriving Care Recommendations}

% TODO: Logic matrix:
% TODO: Low light + High moisture = "Caution: mold risk"
% TODO: High temp + Low water = "Urgent watering needed"


% Visualisierung und Interpretation
\subsection{Visualization and Interpretation}

% TODO: How do you show this to the user? (Traffic light system: Red/Yellow/Green, line charts for weekly trends)
% TODO: Importance of historical data (recognize trends: "Soil dries out faster than before - plant needs repotting")


% Automatisierte Systeme
\subsection{Automated Systems}

% TODO: Brief outlook: Could the system also water automatically? (Control pump based on data)


\section{System Implementation and Validation}

% Hardware-Realisierung und Aufbau
\subsection{Hardware Realization and Setup}

% TODO: Detailed description of the circuit diagram
% TODO: Explain power supply (battery vs. power adapter) and housing (water protection, 3D printing or box)
% TODO: Discuss assembly problems (e.g., solder joints, cable management)


% Software-Architektur und Firmware
\subsection{Software Architecture and Firmware}

% TODO: Explain code structure (Setup, Loop, Deep-Sleep mode)
% TODO: Describe libraries used and how you solved sensor calibration in code (value mapping)


% Ergebnisdarstellung: Das "Plant Up!" Dashboard
\subsection{Results Presentation: The "Plant Up!" Dashboard}

% TODO: Describe the user interface
% TODO: How are alarms displayed? How is history presented?
% TODO: (Screenshots will be inserted here and explained in the text)


% Validierung und Messreihenanalyse
\subsection{Validation and Measurement Series Analysis}

% TODO: The proof: Describe a test run
% TODO: Show a diagram of a 24h measurement and analyze what you see
% TODO: (e.g., "After watering at 14:00, the value increased from 10% to 80%")


\section{Conclusion}

% Zusammenfassung der Ergebnisse
\subsection{Summary of Results}

% TODO: "Plant Up!" works, data is reliably captured


% Bewertung des technologischen Potenzials
\subsection{Evaluation of Technological Potential}

% TODO: Is the system mass-market ready? (Cost per unit vs. benefit)


% Ausblick auf zukünftige Entwicklungen
\subsection{Outlook on Future Developments}

% TODO: What would you do better in version 2.0?
% TODO: (Solar power, AI integration for plant recognition via camera, app integration)


% Roter Faden zum nächsten Thema
\subsection{Connection to the Next Chapter}

% TODO: Transition to the next team member's topic


% Temporary: prevents BibTeX error until we add real citations
\nocite{*}


\chapter[IoT Data Sync in Microservices]{IoT Data Sync in Microservices: Evaluating MQTT vs. REST}
\pagenumbering{arabic}
\setcounter{page}{1}

\section{Introduction}

\subsection{Background and Context}
The integration of the Internet of things (IoT) into our lives has been rapid, especially in the agricultural sector which has started a transformation known as ``Agriculture 4.0'', where data-driven decision making replaces traditional heuristic methods [1]. While industrial has benefited form that shift, a parallel trend is emerging in the consumer sector. Smart Urban gardening [2]. Driven by rapid urbanization and growing societal focus on sustainability, this sector is experiencing significant growth [3]. This surge represents a shift toward the Social Internet of Things (SIoT) [4]. In the SIoT paradigm, objects are capable of establishing social relationships with other objects and humans to foster collaboration[5] However, applying these advanced concepts to consumer grade hardware presents architectural challenges [6]. Unlike industrial systems, consumer IoT devices for plant care must operate in resource constrained environments, reyling on batteries and communicating over congested residential Wi-Fi [7].

\subsection{Case Study: The ``Plant Up!'' Project}
This thesis utilizes the ``Plant Up!'' project as a primary case study to investigate architecural frictions [8]. ``Plant Up!'' is an IoT application designed to gamify the experience of plant care [9]. The system combines hardware sensor units (IoT devices), cloud services and a mobile app to monitor soil moisture, temperature, humidity, light levels and other metrics in real time. The codebase relies on a three tier architecture: edge layer (esp32), the microservices layer and the Application layer [11].

\subsection{Problem Statement}
The design of a real time social plant monitoring system with gamification aspects introduces a conflict between latency, energy efficiency and data consistency [12]. The ``Social'' aspect relies on gamification mechanics that require low latency data transmission [13]. However, the hardware must utilize aggressive power saving states and techniques to be practical for home use [14]. Research indicates that while deep sleep extends battery life, the wake up and following processes cause latency that conflicts with real time requirements [15]. Furthermore, the choice of communication protocol dictates the ``wake up tax'' of the device [16]. Traditional web protocols like HTTP are robust but carry significant header overhead, whereas lightweight protocols like MQTT are designed for exactly that missing efficiency [17]. Additionally, maintaining a consistent view of the system state in a distributed architecture is non trivial, as the CAP theorem dictates trade offs between Consistency, Partition tolerance and Availability [18]

\subsection{Research Question}
To address these challenges, this thesis poses the following primary research question:
How do MQTT and REST compare in a microservices-based architecture for real-time plant monitoring, specifically regarding latency, throughput, and data consistency, when constrained by battery-powered IoT devices [19]?


\section{Theoretical Background}

\subsection{Microservices Architecture (MSA)}
Microservices Architecture (MSA) describes a software design approach in which an application is decomposed into small, autonomous services that communicate through lightweight protocols \cite{ref20}. Each service is responsible for a specific domain and can be deployed and scaled independently. For IoT systems, where hardware events, sensor data ingestion, user interactions and analytics occur asynchronously, this architectural style offers clear advantages in resilience and scalability \cite{ref21}.

The ``Plant Up!'' project follows this principle by organizing its backend into domain-oriented Supabase schemas: the \texttt{user\_schema} manages identities and streak logic, the \texttt{social\_media\_schema} stores posts and plant information, the \texttt{gamification} schema handles XP, quests and rewards, and the \texttt{microcontroller\_schema} stores IoT sensor readings. Each schema acts as an isolated bounded context.

A representative example is the table for environmental sensor readings:

\begin{figure}[H]
    \centering
    \includegraphics[width=0.8\textwidth]{images/Controllers table in microcontroller__schema.png}
    \caption{Controllers table in microcontroller\_schema}
    \label{fig:controllers_table}
\end{figure}

This separation prevents side effects: updating sensor data cannot interfere with social media functionality or gamification services. MSA therefore supports robustness and domain clarity.

\subsection{Core Characteristics of Microservices in IoT}
IoT systems introduce unique requirements that make certain microservice characteristics particularly important. Constrained hardware, intermittent connectivity and asynchronous event streams require an architecture that is tolerant to partial failures and scalable under variable load.

\textbf{Autonomy:}  
Each service must function independently. For example, the gamification subsystem maintains XP and level state regardless of the status of the sensor ingestion pipeline:

\begin{figure}[H]
    \centering
    \includegraphics[width=0.8\textwidth]{images/Gamification player statistics.png}
    \caption{Gamification player statistics}
    \label{fig:gamification_stats}
\end{figure}

\textbf{Loose Coupling:}  
Services communicate through clearly defined data structures. A social media post does not depend on the microcontroller service:

\begin{figure}[H]
    \centering
    \includegraphics[width=0.8\textwidth]{images/Posts table in social__media__schema.png}
    \caption{Posts table in social\_media\_schema}
    \label{fig:posts_table}
\end{figure}

\textbf{Scalability:}  
Sensor bursts occur when multiple devices wake simultaneously. Only the ingestion path needs to scale, not the entire backend.

\textbf{Resilience:}  
Since IoT devices frequently disconnect (deep sleep, Wi-Fi loss), the system must tolerate missing or delayed data. The schema boundaries allow delayed writes without blocking dependent features.

\subsection{JSON Payload Design}
JSON is used throughout Plant Up! for communication between mobile clients, backend services and IoT devices. Because the ESP32-S3 is battery-powered, payload size and structure directly influence energy consumption \cite{ref22}.

A typical payload aligned with the \texttt{Controllers} schema is:

\begin{figure}[H]
    \centering
    \includegraphics[width=0.8\textwidth]{images/Sensor payload structure.png}
    \caption{Sensor payload structure}
    \label{fig:sensor_payload}
\end{figure}

The payload reflects several design principles:
\begin{itemize}
    \item flat JSON structure to avoid deep nesting,
    \item short but descriptive field names to reduce size,
    \item consistent alignment with database schema,
    \item predictable field order for efficient parsing on constrained hardware.
\end{itemize}

Such optimizations reduce transmission time and extend device battery life.

\subsection{Internet of Things (IoT) Constraints and Hardware}
IoT hardware such as the ESP32-S3 operates under constraints that strongly influence backend architecture.

\textbf{1. Power Consumption:}  
Deep sleep drastically reduces energy usage, but each wake cycle incurs overhead due to Wi-Fi reconnection and sensor initialization. This ``wake-up tax'' conflicts with real-time monitoring needs \cite{ref23}.

\textbf{2. Intermittent Connectivity:}  
Home networks produce variable latency. Timestamped readings (\texttt{time}) allow the backend to reconstruct temporal context even with delayed uploads.

\textbf{3. Limited Compute and Memory:}  
Operations like TLS negotiation, JSON serialization and sensor polling must be minimized.

\textbf{4. Sensor Noise \& Calibration:}  
Environmental data such as soil moisture, EC and humidity fluctuate naturally. The backend must treat sensor readings as approximations and potentially smooth or validate them before using them for plant health scoring or gamification.

These constraints justify efficient protocols and eventual consistency strategies.

\subsection{Communication Protocols: MQTT vs. REST}
Communication choices fundamentally shape IoT system performance. In Plant Up!, both REST and MQTT serve different roles.

\textbf{REST (HTTPS):}  
REST is used for structured, authenticated application features such as user data, posts, quests and plant profiles. For example, plant instances are stored as:

\begin{figure}[H]
    \centering
    \includegraphics[width=0.8\textwidth]{images/Plants table.png}
    \caption{Plants table}
    \label{fig:plants_table}
\end{figure}

REST integrates seamlessly with Supabase but introduces overhead due to large HTTP headers and TLS handshakes, which is suboptimal for energy-constrained IoT devices.

\textbf{MQTT:}  
MQTT is a lightweight publish/subscribe protocol optimized for constrained hardware \cite{ref24}. Devices publish compact JSON messages to predefined topics and immediately return to deep sleep. This minimizes active Wi-Fi time and improves battery longevity.

A topic pattern suitable for Plant Up! is:

\begin{figure}[H]
    \centering
    \includegraphics[width=0.8\textwidth]{images/topic pattern for Plant Up! .png}
    \caption{Topic pattern for Plant Up!}
    \label{fig:topic_pattern}
\end{figure}

\textbf{Architectural Implication:}  
Plant Up! adopts a hybrid model:
\begin{itemize}
    \item REST for app features requiring authentication and structured queries,
    \item MQTT for energy-efficient, low-latency sensor data ingestion.
\end{itemize}

\subsection{Data Consistency and the CAP Theorem}
In distributed systems, the CAP theorem states that a system cannot simultaneously provide Consistency (C), Availability (A) and Partition Tolerance (P) \cite{ref25}. Since IoT systems must assume network partitions due to intermittent connectivity, Partition Tolerance is unavoidable.

Plant Up! prioritizes Availability and Partition Tolerance (AP), accepting that sensor readings may arrive late and the system will become consistent over time.

Upon reconnection from deep sleep, readings are inserted into the \texttt{Controllers} table:

\begin{figure}[H]
    \centering
    \includegraphics[width=0.8\textwidth]{images/Insert example for sensor readings.png}
    \caption{Insert example for sensor readings}
    \label{fig:insert_example}
\end{figure}

Analytics and gamification services operate on this eventually consistent data. This ensures:
\begin{itemize}
    \item the mobile app remains responsive,
    \item no sensor data is lost during connectivity gaps,
    \item gamification logic (e.g., XP, quests) updates when sufficient data is available.
\end{itemize}

This trade-off aligns with the practical constraints of consumer IoT devices.


\section{Plant Up! System Architecture and Implementation}

\subsection{System Overview and Requirements}
The ``Plant Up!'' system integrates IoT hardware, microservices and a mobile application into a unified platform designed to support real-time plant monitoring and social interaction. The primary system objective is to collect environmental sensor data from consumer-grade hardware, synchronize it across distributed cloud services and present it to users in a gamified, socially engaging interface.

To achieve this, the system must satisfy three key requirements:
\begin{itemize}
    \item \textbf{Low-latency sensor synchronization:} Environmental data must be transmitted and processed fast enough to provide timely feedback on plant health.
    \item \textbf{Energy-efficient operation:} IoT devices must conserve battery life through deep sleep cycles, lightweight payloads and minimal active radio time.
    \item \textbf{Distributed consistency:} Since data is stored across multiple domain-specific schemas, the system must tolerate intermittent connectivity while ensuring eventual consistency.
\end{itemize}

These requirements shape the architectural decisions in both hardware and microservices. The system therefore follows a multi-layered design consisting of an edge node, a cloud microservices layer and a client application layer.

\subsection{Hardware Layer: The Edge Node}
The hardware layer consists of an ESP32-S3 microcontroller equipped with sensors for temperature, humidity, light intensity, soil moisture and electrical conductivity. This configuration enables comprehensive monitoring of plant health. The device operates under strict energy constraints and therefore relies heavily on deep sleep modes. Upon waking, it performs three tasks:

\begin{enumerate}
    \item Reads environmental sensors.
    \item Serializes the data into a compact JSON structure.
    \item Sends the payload to the backend via MQTT or REST before returning to deep sleep.
\end{enumerate}

The JSON payload corresponds to the structure of the \texttt{Controllers} table in the \texttt{microcontroller\_schema}:

\begin{figure}[H]
    \centering
    \includegraphics[width=0.8\textwidth]{images/Sensor payload structure.png}
    \caption{JSON payload sent by the edge node}
    \label{fig:edge_json_payload}
\end{figure}

The corresponding Supabase table is:

\begin{figure}[H]
    \centering
    \includegraphics[width=0.8\textwidth]{images/Controllers table in microcontroller__schema.png}
    \caption{Controllers table receiving IoT data}
    \label{fig:controllers_table_rx}
\end{figure}

By timestamping each measurement, the backend can reconstruct time series data even when devices experience connectivity delays, enabling eventual consistency.

\subsection{Microservice Architecture Design}
The backend adopts a microservices-inspired architecture grounded in Supabase schemas. Each schema represents a bounded context aligned with a specific domain:

\begin{itemize}
    \item \textbf{user\_schema}: Stores user profiles, streak data and virtual currency.
    \item \textbf{social\_media\_schema}: Manages posts, comments, plants and plant metadata.
    \item \textbf{microcontroller\_schema}: Stores IoT sensor measurements and device associations.
    \item \textbf{gamification}: Contains quests, user quest progress and XP statistics.
\end{itemize}

This separation allows backend services to evolve independently and prevents cross-domain interference. For example, the creation of a post does not impact sensor ingestion, and a device update does not affect quest completion logic.

An example of domain separation is evident in the \texttt{Plants} table:

\begin{figure}[H]
    \centering
    \includegraphics[width=0.8\textwidth]{images/Plants table.png}
    \caption{Plants table in social\_media\_schema}
    \label{fig:plants_table_arch}
\end{figure}

This table links user-owned plants to ideal environmental parameters in \texttt{Plant\_data}, enabling health comparisons by other services.

\subsection{Real-Time Data Synchronization Mechanism}
Real-time synchronization in ``Plant Up!'' requires balancing two competing goals: minimizing battery usage on the edge node while providing timely updates to the microservices layer. Two communication mechanisms are evaluated:

\begin{itemize}
    \item \textbf{REST (HTTPS)} offers structured, authenticated, synchronous transmission suitable for user-driven interactions but incurs significant overhead.
    \item \textbf{MQTT} provides lightweight, publish/subscribe semantics with lower transmission cost, making it more suitable for the ESP32-S3.
\end{itemize}

Sensor data is synchronized using a hybrid approach. IoT controllers publish sensor payloads through MQTT for efficiency. The backend processes the incoming data and stores it in \texttt{microcontroller\_schema.Controllers}. User-facing services such as the mobile app retrieve the aggregated data via REST.

A typical ingestion operation is represented by:

\begin{figure}[H]
    \centering
    \includegraphics[width=0.8\textwidth]{images/Insert example for sensor readings.png}
    \caption{Insert operation for real-time sensor synchronization}
    \label{fig:insert_op_sync}
\end{figure}

Since devices may reconnect sporadically, synchronization follows an eventually consistent model rather than strict ordering guarantees.

\subsection{Social Gamification Logic Implementation}
Gamification is integrated into the Plant Up! experience to motivate sustained engagement. The gamification subsystem relies on three core tables:

\begin{itemize}
    \item \textbf{Quests}: Defines daily and weekly goals.
    \item \textbf{User\_quests}: Tracks per-user quest progression.
    \item \textbf{player\_stats}: Stores XP and level totals.
\end{itemize}

For example, the quests table is defined as:

\begin{figure}[H]
    \centering
    \includegraphics[width=0.8\textwidth]{images/Quests table in gamification schema.png}
    \caption{Quests table in gamification schema}
    \label{fig:quests_table}
\end{figure}

When a user completes an action, such as watering a plant or posting an update, the backend increments their quest progress:

\begin{figure}[H]
    \centering
    \includegraphics[width=0.8\textwidth]{images/User quest progression.png}
    \caption{User quest progression}
    \label{fig:user_quest_progression}
\end{figure}

Once the target count is met, XP is awarded:

\begin{figure}[H]
    \centering
    \includegraphics[width=0.8\textwidth]{images/Awarding XP to player_stats.png}
    \caption{Awarding XP to player\_stats}
    \label{fig:xp_award}
\end{figure}

This modular design ensures that the gamification logic remains independent of the sensor ingestion pipeline, social media features and plant data system.


\section{Experimental Evaluation of MQTT vs. REST}

\subsection{Introduction to the Experiment}
To provide a rigorous answer to the research question—How do MQTT and REST compare in a microservices-based architecture for real-time plant monitoring?—this thesis employs an empirical experimental approach. The ``Plant Up!'' system, as detailed in Chapter 3, serves as the testbed. The evaluation focuses on quantifying the three critical performance vectors defined in the problem statement: Latency, Throughput, and Energy Efficiency.

This chapter details the experimental setup, the measurement methodology, and the specific scenarios designed to stress-test the protocols under conditions mimicking a real-world urban gardening environment (e.g., unstable Wi-Fi, battery constraints).

\subsection{Experimental Setup}
The testbed is constructed to isolate the communication protocol as the single independent variable. Both the hardware (Edge) and the backend (Cloud) remain constant, with only the application layer transport mechanism toggling between MQTT (v3.1.1) and REST (HTTP/1.1).

\subsubsection{Hardware Configuration (Edge Layer)}
The experiments utilize the ESP32-S3-DevKitC-1, selected for its relevance to the ``Plant Up!'' production specification.
\begin{itemize}
    \item \textbf{Microcontroller}: ESP32-S3 (Xtensa\textregistered\ 32-bit LX7 dual-core, 240 MHz).
    \item \textbf{Network Interface}: Integrated 2.4 GHz Wi-Fi (802.11 b/g/n).
    \item \textbf{Power Measurement}: Nordic Semiconductor Power Profiler Kit II (PPK2), set to ``Source Meter'' mode with a sampling rate of 100ksps (kilo-samples per second) to capture transient current spikes during Wi-Fi transmission.
    \item \textbf{Sensors}: Simulated sensor data is used during load testing to ensure deterministic payload sizes, eliminating variance caused by sensor read times.
\end{itemize}

\subsubsection{Backend Environment (Cloud Layer)}
The microservices backend is hosted on a local Kubernetes cluster (Minikube) to eliminate internet service provider (ISP) jitter from the latency measurements.
\begin{itemize}
    \item \textbf{Broker}: Eclipse Mosquitto (v2.0.11) deployed as a Docker container.
    \item \textbf{API Gateway}: NGINX (v1.21) acting as the reverse proxy for REST requests.
    \item \textbf{Database}: Supabase (v2.0) for time-series storage.
    \item \textbf{Network Emulation}: \texttt{tc} (Traffic Control) is used on the Linux host to simulate packet loss (2-5\%) and added latency (50-100ms), replicating poor residential Wi-Fi signal strength (-80dBm).
\end{itemize}

The following infrastructure configuration ensures both protocols are available simultaneously for A/B testing:
% Image removed


\subsection{Methodology and Metrics}

\subsubsection{Metric 1: End-to-End Latency ($L_{e2e}$)}
Latency is defined as the time elapsed between the generation of a sensor reading at the Edge Node ($t_{gen}$) and its persistence in the Supabase database ($t_{ack}$).
\[ L_{e2e} = t_{ack} - t_{gen} \]
To measure this accurately without relying on un-synchronized clocks (clock drift between ESP32 and Server), we utilize a ``Round Trip Time'' (RTT) approach for the benchmark. The ESP32 sends a message and waits for an application-layer acknowledgement from the server.
\begin{itemize}
    \item \textbf{MQTT}: Time from \texttt{publish()} to arrival of \texttt{PUBACK} (QoS 1).
    \item \textbf{REST}: Time from \texttt{http.POST()} to return of HTTP 200 OK.
\end{itemize}
Note: In the REST implementation, \texttt{http.begin()} often initiates the TCP handshake, heavily penalizing the latency score if Keep-Alive is not active.

\begin{center}
    \includegraphics[width=0.8\textwidth]{images/MQTT RTT Measurement (PUBACK timing).png}
    \captionof{figure}{MQTT RTT Measurement (PUBACK timing)}
    \label{fig:temp_image_rtt}
\end{center}
\subsubsection{Metric 2: Throughput and Congestion}
Throughput is evaluated by increasing the message frequency ($f_{msg}$) from 1 Hz to 100 Hz. The metric is Successful Messages Per Second (SMPS).
This test simulates a ``Broadcast Event'' where the ``Social Service'' might request immediate status updates from all plants in a Guild (e.g., during a ``Watering Party'' game event). We observe the point at which packet loss exceeds 1\% or the ESP32's queue overflows.

\subsubsection{Metric 3: Energy Consumption}
Using the DEBUG\_PIN triggers the Logic Analyzer to capture the exact current draw during the transmission window.
\begin{center}
    \includegraphics[width=0.8\textwidth]{images/GPIO Toggle for Measurement Trigger.png}
    \captionof{figure}{GPIO Toggle for Measurement Trigger}
    \label{fig:temp_image_gpio}
\end{center}

To evaluate Data Consistency, we induce network partitions (simulating a user walking out of Wi-Fi range with the portable sensor unit).

\textbf{Scenario}: The device generates 100 messages while disconnected.
\begin{itemize}
    \item \textbf{REST Behavior}: The \texttt{HTTPClient} returns connection errors. We measure how many messages are lost vs. how many are successfully buffered in the ESP32's limited RAM (Ring Buffer) and queued for reconnection.
    \item \textbf{MQTT Behavior}: We test QoS 1 and QoS 2 with Persistent Sessions (CleanSession=false). The Broker should queue messages destined for the subscriber, but the Edge Node must also queue messages destined for the Cloud.
\end{itemize}

This evaluation specifically looks at the implementation of the Outbox Pattern on the embedded device, which is crucial for the ``Eventual Consistency'' required by the CAP theorem analysis in Chapter 2.4.

\begin{center}
    \includegraphics[width=0.8\textwidth]{images/Outbox Pattern Buffering on ESP32.png}
    \captionof{figure}{Outbox Pattern Buffering on ESP32}
    \label{fig:temp_image_outbox}
\end{center}



%--------------------------------------------------------------------------
% Anhang
%--------------------------------------------------------------------------
   
 
  \addcontentsline{toc}{chapter}{Anhang}
    \phantomsection
	\addcontentsline{toc}{section}{Abbildungsverzeichnis}
	\listoffigures
	\cleardoublepage
	\phantomsection
	
	\addcontentsline{toc}{section}{Literaturverzeichnis}
	\bibliographystyle{IEEEtran}
	\bibliography{references}
	
	\cleardoublepage
	\phantomsection

    \cleardoublepage
    \phantomsection
 
\end{document}