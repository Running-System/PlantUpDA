%!TEX root = ../../Diplomarbeit.tex
\section{Theoretical Foundations of Gamification}

\subsection{Definition of Gamification}

Gamification is a term that defines how game design elements and mechanics are integrated into non-game contexts while the main activity remains a non-game.
The goal of introducing these elements is to increase user engagement and motivation by making otherwise boring or routine tasks more appealing \cite{gablerGamification, grendelGamification}.

\subsection{Difference between Gamification, Serious Games, and Games for Entertainment}

It is important to note that gamification is not the same as serious games or games for entertainment.
Although the terms sound similar, they describe different concepts.
Gamification refers to the integration of game elements into non-game contexts in order to increase user engagement and motivation.
Serious games are complete games that are explicitly designed for a primary purpose other than entertainment, such as education, training, or simulation.
A common example is a flight simulator used for pilot training.
Games for entertainment, on the other hand, are produced solely for enjoyment, without any explicit educational or functional objective \cite{MDPI}.

\subsection{Why Gamification Works in Non-Game Contexts}

While many people are skeptical about the effectiveness of gamification, research and practice have shown that it can be an effective strategy for enhancing user engagement and motivation.
This raises the questions: Why and how?

According to the Haufe Akademie, the effectiveness of gamification lies deep within the human psyche.
It addresses the basic needs of autonomy, competence, and social relatedness, which form the foundation of human motivation.
How gamification achieves this can be explained using three central mechanisms.
\begin{itemize}
    \item \textbf{1:} Gamification integrates the human need for autonomy, competence, and social relatedness not merely through external rewards, but by actively supporting these needs.
    \item \textbf{2:} It redefines user interaction with a system. Instead of removing all difficult or boring tasks to maximize efficiency, gamification introduces voluntary challenges that encourage people to improve themselves. These tasks are transformed into engaging activities in which progress is clearly visualized and supported by immediate feedback mechanisms such as points and badges.
    \item \textbf{3:} Gamification uses social interaction to build a sense of community. Features such as cooperative challenges and leaderboards satisfy the desire to be noticed by others, increasing attachment to the activity through competitive or collaborative motivation \cite{haufeGamification}.
\end{itemize}

\subsection{Motivation Theory}

To understand why gamification can be effective in practice, it is necessary to examine the foundations of human motivation.
Motivation is the psychological force that initiates, guides, and sustains goal-directed behavior.
In the context of gamification, a key distinction is made between engaging in an activity because it is inherently rewarding (intrinsic motivation) and engaging in it to obtain an external outcome (extrinsic motivation) \cite{bandhuMotivation}.

Motivation should not be regarded as a monolithic construct, as it exists on a spectrum.
The two primary types relevant to gamification are intrinsic and extrinsic motivation.

\subsubsection{Intrinsic Motivation (The Engine of Engagement)}

Intrinsic motivation refers to engaging in an activity because it is satisfying, enjoyable, or meaningful and is considered essential for long-term engagement \cite{bandhuMotivation2024}.
According to Self-Determination Theory (SDT) \cite{sdtOrgTheory}, intrinsic motivation is supported when three basic psychological needs are fulfilled:
\begin{itemize}
    \item \textbf{Autonomy:} The perception of choice and self-direction.
    \item \textbf{Competence:} The experience of mastery and effectiveness.
    \item \textbf{Relatedness:} The feeling of belonging and social connection.
\end{itemize}
When these needs are satisfied, individuals demonstrate increased curiosity, creativity, and persistence \cite{bandhuMotivation2024}.

\subsubsection{Extrinsic Motivation (The Spark)}

Extrinsic motivation involves performing an activity in order to receive a reward or avoid negative consequences.
Common gamification elements such as points, badges, and leaderboards are examples of extrinsic motivators \cite{bandhuMotivation2024}.
\begin{itemize}
    \item \textbf{Controlling Extrinsic Motivation:} When rewards are perceived as coercive or punitive, they can undermine intrinsic motivation, resulting in short-term compliance but reduced long-term engagement. A typical example is a streak-based system that pressures users to perform regularly, which can lead to stress and burnout.
    \item \textbf{Autonomy-Supportive Extrinsic Motivation:} When rewards provide informational feedback or signal competence, they can be internalized and reinforce intrinsic motivation. Feedback such as positive reinforcement for progress or improvement is more effective than obligation-based incentives.
\end{itemize}

Effective gamification strategies use extrinsic rewards not as the primary objective, but as a means of supporting intrinsic motivation \cite{haufeGamification, bandhuMotivation2024}.

\subsubsection{Self-Determination Theory (SDT) in Gamification}

Self-Determination Theory establishes a framework for human motivation based on three fundamental psychological needs: autonomy, competence, and relatedness.
Meeting these needs fosters self-determined motivation, which drives improved performance, persistence, and creativity.
Failing to support them can significantly reduce motivation and well-being \cite{bandhuMotivation2024, sdtOrgTheory}.

\subsubsection{Supporting Autonomy}

Users should experience a sense of control over their actions.
\begin{itemize}
    \item \textbf{Design Patterns:} Optional tasks, multiple paths to success, and customization options \cite{bandhuMotivation2024}.
\end{itemize}

\subsubsection{Supporting Competence}

Users require clear feedback that demonstrates progress and skill development.
\begin{itemize}
    \item \textbf{Design Patterns:} Experience points, progress indicators, adaptive difficulty, and immediate feedback loops \cite{bandhuMotivation2024}.
\end{itemize}

\subsubsection{Supporting Relatedness}

Users are more engaged when they feel socially connected.
\begin{itemize}
    \item \textbf{Design Patterns:} Social feeds, cooperative goals, peer comparison, and mentorship features \cite{bandhuMotivation2024}.
\end{itemize}

\subsubsection{The Role of Extrinsic Rewards}

Gamification is not merely about adding rewards, but about how they are implemented.
Extrinsic motivators can be powerful drivers of engagement but may reduce creativity and encourage mechanical behavior.
This can result in retention driven by obligation rather than enjoyment \cite{gamifyMotivation}.
When used carefully, extrinsic motivators can still provide short-term motivational boosts that foster intrinsic interests such as mastery or competition \cite{gamifyMotivation, bandhuMotivation2024}.

\subsubsection{Rewards that Support Intrinsic Motivation}

Intrinsic motivators are linked to a personal desire to master, explore, and enjoy an activity.
They promote positive engagement and long-term commitment when implemented as informational feedback rather than control mechanisms.
\begin{itemize}
    \item \textbf{Badges as Recognition:} Indicators of achievement and mastery.
    \item \textbf{Experience Points as Feedback:} Visualization of personal growth.
    \item \textbf{Virtual Currency for Customization:} Enabling autonomy and self-expression.
    \item \textbf{Titles as Identity Markers:} Reinforcing competence and long-term commitment \cite{bandhuMotivation2024}.
\end{itemize}

\subsubsection{Rewards that Undermine Intrinsic Motivation}

Rewards can become counterproductive when they are perceived as controlling or meaningless.
In smart gardening applications, where engagement must be sustained over long periods despite repetitive tasks, this risk is particularly relevant.
\begin{itemize}
    \item \textbf{Trivial Action Rewards:} Over-rewarding insignificant actions.
    \item \textbf{Punitive Streak Systems:} Excessive penalties that induce stress.
    \item \textbf{Context-Free Leaderboards:} Rankings without fairness or relevance.
    \item \textbf{Motivational Displacement:} When rewards replace genuine interest in the activity itself \cite{bandhuMotivation2024}.
\end{itemize}

\subsubsection{Attribution Theory}

Attribution theory examines how individuals explain success and failure.
In gamified systems, it is important that users attribute outcomes to their own effort and strategies rather than to luck or fixed ability.
\begin{itemize}
    \item \textbf{Effort-Based Feedback:} Emphasizing controllable factors increases motivation.
    \item \textbf{Failure as Learning Opportunity:} Errors should be framed as temporary and improvable.
\end{itemize}

\subsubsection{Expectancy-Value Theory}

Before engaging in an activity, users implicitly evaluate two questions: Can I succeed and is this worth my effort?
\begin{enumerate}
    \item \textbf{Expectancy:} Clear goals and visible progress increase perceived attainability.
    \item \textbf{Value:} The task must be meaningful, useful, or personally relevant \cite{bandhuMotivation2024}.
\end{enumerate}

\subsection{Summary: Principles of Motivational Gamification}

Effective gamification systems do not replace intrinsic motivation with rewards.
Instead, they employ extrinsic elements to support autonomy, competence, and relatedness.
When rewards are perceived as informational rather than controlling, they foster engagement, learning, and sustainable behavioral change.

\subsection{Engagement and Habit Formation}

\begin{itemize}
    \item Feedback loops
    \item Short-term versus long-term engagement
    \item Daily routines and behavioral reinforcement
\end{itemize}
