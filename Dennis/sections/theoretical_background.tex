%!TEX root = ../../Diplomarbeit.tex
\section{Theoretical Foundations of Gamification}

\subsection{Definition of Gamification}

Gamification, a term that defines how game design elements and mechanics are integrated to non-game contexts, while the main activity remains a non-game.
The goal of introducing these elements is to increase user engagement and motivation by making otherwise boring or routine tasks more appealing \cite{gablerGamification, grendelGamification}.

\subsection{Difference between Gamification, Serious Games, and Games for Entertainment}

It is important to take note that gamification is not the same as serious games or games for entertainment.
Although the terms sound the same, they describe different concepts.
While gamification refers to the integration of game elements into non-game contexts in order to increase user engagement and motivation, serious games are complete games that are explicitly designed for a primary purpose other than entertainment, such as education, training, or simulation. 
A common example is a flight simulator used for pilot training.
On the other hand games for entertainment are only produced for enjoyment, without any explicit educational or functional objective \cite{MDPI}.

\subsection{Why Gamification Works in Non-Game Contexts}

While many people are skeptical about the effectiveness of gamification, research and practice have shown that it can be an effective strategy for enhancing user engagement and motivation. 
This raises the questions: 'Why and how?'.

According to the Haufe Akademie, the effectiveness of gamification lies deep in the human psyche, it addresses the basic needs of autonomy, competence, and social relatedness, which are the foundation of human motivation.
How gamification achieves this can be explained using three central mechanisms.
\begin{itemize}
    \item \textbf{1:} Gamification integrates the human need for autonomy, competence, and social relatedness not merely through external rewards, but by supporting these needs.
    \item \textbf{2:} It redefines user interaction with a system. Instead of removing all hard and boring tasks to maximize efficiency, gamification introduces voluntary challenges that encourage people to improve themselves. These tasks are transformed into engaging activities in which progress is clearly visualized and displayed with immediate feedback mechanisms such as points, badges and so on. 
    \item \textbf{3:} Gamification uses social interaction to build a sense of community. Features such as cooperative challenges and leaderboards satisfy the desire of being noticed by others, thereby people feel more attached to the activity, by getting a competitive feeling.  \cite{haufeGamification}.

\subsection{Motivation Theory}

To understand why gamification can be effective in practice, it is necessary to examine the foundations of human motivation. Motivation is the psychological force that initiates, guides, and sustains goal-directed behavior. In the context of gamification, a key distinction is made between engaging in an activity because it is inherently rewarding (intrinsic motivation) and engaging in it to obtain an external outcome (extrinsic motivation) \cite{bandhuMotivation}.

Motivation should not be thought about as a monolithic construct because it exists on a spectrum. The two primary types, that are relevant to gamification, are intrinsic- and extrinsic motivation.

\subsubsection{Intrinsic Motivation (The Engine of Engagement)}

Intrinsic motivation refers to engaging in an activity because it is satisfying, enjoyable, or meaningful, which is considered essential for long-term engagement \cite{bandhuMotivation2024}. 
According to Self-Determination Theory (SDT) \cite{sdtOrgTheory}, intrinsic motivation is supported when three basic psychological needs are fulfilled:
\begin{itemize}
    \item \textbf{Autonomy:} The perception of choice and self-direction.
    \item \textbf{Competence:} The experience of mastery and effectiveness.
    \item \textbf{Relatedness:} The feeling of belonging and social connection.
\end{itemize}
When these needs are satisfied, individuals demonstrate increased curiosity, creativity, and persistence \cite{bandhuMotivation2024}.

\subsubsection{Extrinsic Motivation (The Spark)}

Extrinsic motivation involves performing an activity in order to receive a reward or avoid negative consequences. Common gamification elements such as points, badges, and leaderboards are examples of extrinsic motivators \cite{bandhuMotivation2024}.
\begin{itemize}
    \item \textbf{Controlling Extrinsic Motivation:} When rewards are perceived as coercive or punitive, they can undermine intrinsic motivation, resulting in short-term compliance but reduced long-term engagement. A typical example would be a streak-based system, where the user needs to accomplish something regularly to keep his streak going. This can be counterproductive as it puts pressure on the user to perform and can lead to burnout.
    \item \textbf{Autonomy-Supportive Extrinsic Motivation:} On the contrary, when rewards provide informational feedback or signal competence, they can be internalized and reinforce intrinsic motivation. Instead of 'You need to do this!' it is more effective to say 'You did this well!' or 'You are getting better at this!', mostly this is done by just handing out rewards for accomplishments, while not making them feel like a chore.
\end{itemize}

Effective gamification strategies use extrinsic rewards not as the primary objective, but as a means of supporting intrinsic motivation \cite{haufeGamification, bandhuMotivation2024}.

\begin{figure}[H]
    \centering
    \includegraphics[width=0.8\textwidth]{images/motivation_types.png}
    \caption{Motivation to Learn: Intrinsic vs. Extrinsic \cite{bandhuMotivation2024}}
    \label{fig:motivation_types}
\end{figure}

\subsubsection{Self-Determination Theory (SDT) in Gamification}
Self-Determination Theory (SDT) establishes a framework for human motivation based on three fundamental psychological needs: autonomy, competence, and relatedness.
Meeting these needs fosters self-determined motivation, which drives enhanced performance, persistence, and creativity. 
In contrast, failing to support them can significantly diminish motivation and well-being \cite{bandhuMotivation2024, sdtOrgTheory}.

\subsubsection{Supporting Autonomy}
Users should experience a sense of control over their actions.
\begin{itemize}
    \item \textbf{Design Patterns:} Optional tasks, multiple paths to success, and customization options. \cite{bandhuMotivation2024}
\end{itemize}

\subsubsection{Supporting Competence}
Users need clear feedback that demonstrates progress and skill development.
\begin{itemize}
    \item \textbf{Design Patterns:} Experience points, progress indicators, difficulty scaling, and immediate feedback loops. \cite{bandhuMotivation2024}
\end{itemize}

\subsubsection{Supporting Relatedness}
Users are more engaged when they feel socially connected.
\begin{itemize}
    \item \textbf{Design Patterns:} Social feeds, cooperative goals, peer comparison, and mentorship features. \cite{bandhuMotivation2024}
\end{itemize}

\begin{figure}[H]
    \centering
    \includegraphics[width=1.0\textwidth]{images/sdt_model.png}
    \caption{The Self-Determination Continuum of Motivation \cite{bandhuMotivation2024}}
    \label{fig:sdt_model}
\end{figure}

\subsubsection{The Role of Extrinsic Rewards}

Gamification is not merely about adding rewards, but about how they are implemented. Extrinsic motivators, which can be internal (rewards) or external (prizes) to the system, are known outcomes that drive play. While they can be strong drivers, they often come at the expense of creative problem-solving and can lead to mechanical gameplay (grinding). This results in retention (negative engagement) rather than enchantment (positive engagement) \cite{gamifyMotivation}. However, extrinsic motivators can still be effective by providing short-term motivational boosts that may eventually foster intrinsic interests such as competition or mastery \cite{gamifyMotivation, bandhuMotivation2024}.

\subsubsection{Rewards that Support Intrinsic Motivation}
Intrinsic motivators are linked to a personal desire to master, explore, and enjoy the activity. Unlike extrinsic rewards, they generally lead to enchantment, where users feel a sense of ``positive engagement'' and are more likely to evangelize the system \cite{gamifyMotivation}. Effective rewards in this category function as informational feedback rather than control mechanisms.
\begin{itemize}
    \item \textbf{Badges as Recognition:} Indicators of achievement and mastery.
    \item \textbf{Experience Points as Feedback:} Visualization of personal growth.
    \item \textbf{Virtual Currency for Customization:} Enabling autonomy and self-expression.
    \item \textbf{Titles as Identity Markers:} Reinforcing competence and long-term commitment. \cite{bandhuMotivation2024}
\end{itemize}

\subsubsection{Rewards that Undermine Intrinsic Motivation}
Rewards can become counterproductive when they are perceived as controlling or meaningless.
In the context of smart gardening applications, these motivational mechanisms are particularly relevant, as user engagement must be sustained over long periods despite repetitive tasks such as monitoring or maintenance.
\begin{itemize}
    \item \textbf{Trivial Action Rewards:} Over-rewarding insignificant actions.
    \item \textbf{Punitive Streak Systems:} Excessive penalties that induce stress.
    \item \textbf{Context-Free Leaderboards:} Rankings without fairness or relevance.
    \item \textbf{Motivational Displacement:} When rewards replace interest in the activity itself \cite{bandhuMotivation2024}.
\end{itemize}

\subsubsection{Attribution Theory}

Attribution theory examines how individuals explain success and failure. In gamified systems, it is important that users attribute outcomes to their own effort and strategies rather than to luck or fixed ability.
\begin{itemize}
    \item \textbf{Effort-Based Feedback:} Emphasizing controllable factors increases motivation.
    \item \textbf{Failure as Learning Opportunity:} Errors should be framed as temporary and improvable.
\end{itemize}

\subsubsection{Expectancy-Value Theory}

Before engaging in an activity, users implicitly evaluate two questions: ``Can I succeed?'' (expectancy) and ``Is this worth my effort?'' (value).
\begin{enumerate}
    \item \textbf{Expectancy:} Clear goals and visible progress increase perceived attainability.
    \item \textbf{Value:} The task must be meaningful, useful, or personally relevant \cite{bandhuMotivation2024}.
\end{enumerate}

\subsection{Summary: Principles of Motivational Gamification}

Effective gamification systems do not replace intrinsic motivation with rewards. Instead, they employ extrinsic elements to support autonomy, competence, and relatedness. When rewards are perceived as informational rather than controlling, they foster engagement, learning, and sustainable behavioral change.

\subsection{Engagement and Habit Formation}
\begin{itemize}
    \item Feedback loops
    \item Short-term versus long-term engagement
    \item Daily routines and behavioral reinforcement
\end{itemize}
