%!TEX root = ../../Diplomarbeit.tex
\section{Theoretical Foundations of Gamification}

\subsection{Definition of Gamification}

Gamification defines how game design elements and mechanics are integrated into non-game contexts while the main activity remains a non-game.
The goal of introducing these elements is to increase user engagement and motivation by making otherwise boring or routine tasks more appealing \cite{gablerGamification, grendelGamification}.

\subsection{Difference between Gamification, Serious Games, and Games for Entertainment}

It is important to note that gamification is not the same as serious games or games for entertainment.
Although the terms sound similar, they describe different concepts.
Gamification refers to the integration of game elements into non-game contexts in order to increase user engagement and motivation.
Serious games are complete games that are explicitly designed for a primary purpose other than entertainment, such as education, training, or simulation.
For example, a flight simulator used for pilot training or a surgery simulation used for medical training.
Games for entertainment, on the other hand, are produced only for enjoyment, without any educational or functional objective \cite{MDPI}.

\subsection{Why Gamification Works in Non-Game Contexts}

While many people question the effectiveness of gamification, research has shown that it can be an effective strategy for enhancing user engagement and motivation.
This raises the questions: Why and how?

According to the Haufe Akademie, the effectiveness of gamification lies deep within the human psyche.
It addresses the basic needs of autonomy, competence, and social relatedness, which form the foundation of human motivation.
How gamification achieves this can be explained using three central mechanisms.
\begin{itemize}
    \item \textbf{1:} Gamification integrates the human need for autonomy, competence, and social relatedness not merely through external rewards, but by actively supporting these needs.
    \item \textbf{2:} It redefines user interaction with a system. Instead of removing all difficult or boring tasks to maximize efficiency, gamification introduces voluntary challenges that encourage people to improve themselves. These tasks are transformed into engaging activities in which progress is clearly visualized and supported by immediate feedback mechanisms such as points and badges.
    \item \textbf{3:} Gamification uses social interaction to build a sense of community. Features such as cooperative challenges and leaderboards satisfy the desire to be noticed by others, increasing attachment to the activity through competitive or collaborative motivation \cite{haufeGamification}.
\end{itemize}

\subsection{Motivation Theory}

To understand why gamification can be effective in practice, it is necessary to examine the foundations of human motivation.
Motivation is the psychological force that initiates, guides, and sustains goal-directed behavior.
In the context of gamification, a key distinction is made between engaging in an activity because it is inherently rewarding (intrinsic motivation) and engaging in it to obtain an external outcome (extrinsic motivation) \cite{bandhuMotivation2024}.


\subsubsection{Intrinsic Motivation (The Engine of Engagement)}

Intrinsic motivation refers to engaging in an activity because it is satisfying, enjoyable, or meaningful and is considered essential for long-term engagement \cite{bandhuMotivation2024}.
According to Self-Determination Theory (SDT) \cite{sdtOrgTheory}, intrinsic motivation is supported when three basic psychological needs are fulfilled:  Autonomy, Competence, and Relatedness.

When these needs are satisfied, individuals demonstrate increased curiosity, creativity, and persistence \cite{bandhuMotivation2024}.
\subsubsection{Self-Determination Theory (SDT) in Gamification}

Self-Determination Theory establishes a framework for human motivation based on three fundamental psychological needs: autonomy, competence, and relatedness.
Meeting these needs fosters self-determined motivation, which drives improved performance, persistence, and creativity.
Failing to support them can significantly reduce motivation and well-being \cite{bandhuMotivation2024, sdtOrgTheory}.

\subsubsection{Supporting Autonomy}

Users should experience a sense of control over their actions.
This is achievable with integrating optional tasks, multiple paths to success, and customization options \cite{bandhuMotivation2024}.


\subsubsection{Supporting Competence}

Users require clear feedback that demonstrates progress and skill development, so that they can feel a sense of accomplishment and competence.
It is mostly done by integrating experience points, progress indicators, adaptive difficulty, and immediate feedback loops \cite{bandhuMotivation2024}.

\subsubsection{Supporting Relatedness}

Users are more engaged when they feel socially connected, hence integrating social feeds, cooperative goals, peer comparison, and mentorship features is a go to strategy \cite{bandhuMotivation2024}.

\subsubsection{Extrinsic Motivation (The Spark)}

Extrinsic motivation involves performing an activity in order to receive a reward or avoid negative consequences.
Common gamification elements such as points, badges, and leaderboards are examples of extrinsic motivators \cite{bandhuMotivation2024}.
\begin{itemize}
    \item \textbf{Controlling Extrinsic Motivation:} When rewards are perceived as chore-like, they can reduce intrinsic motivation, resulting in short-term compliance but reduced long-term engagement. A typical example is a streak-based system that pressures users to perform regularly, which can lead to stress and burnout.\cite{bandhuMotivation2024}.
    \item \textbf{Autonomy-Supportive Extrinsic Motivation:} When rewards provide informational feedback or signal competence, they can be internalized and reinforce intrinsic motivation. Feedback such as positive reinforcement for progress or improvement is more effective than obligation-based incentives.\cite{bandhuMotivation2024}.
\end{itemize}

Effective gamification strategies use extrinsic rewards not as the primary objective, but as a means of supporting intrinsic motivation \cite{haufeGamification, bandhuMotivation2024}.



\subsubsection{The Role of Extrinsic Rewards}

Gamification is not merely about adding rewards, but about how they are implemented.
Extrinsic motivators can be powerful drivers of engagement but may reduce creativity and encourage mechanical behavior.
This can result in retention driven by obligation rather than enjoyment \cite{gamifyMotivation}.
When used carefully, extrinsic motivators can still provide short-term motivational boosts that foster intrinsic interests such as mastery or competition \cite{gamifyMotivation, bandhuMotivation2024}.

\subsubsection{Rewards that Support Intrinsic Motivation}

Motivators that support intrinsic motivation are linked to a personal desire to master, explore, and enjoy an activity.
They are the main motivator for long-term commitment, when implemented right, trying to keep them as informational feedback rather than controlling mechanisms is the key to succeeding \cite{gamifyMotivation}
Often those are accomplisdhed by integrating, badges, that indicate achievement and mastery, experience points, virtual currency for customization and titles \cite{gamifyMotivation}.

\subsubsection{Rewards that Undermine Intrinsic Motivation}

Some rewards can be counterproductive when they seem controlling or worthless. \cite{gamifyMotivation}
Such as over-rewarding insignificant actions,excessive penalties that induce stress, rankings without fairness or relevance, and when rewards replace genuine interest in the activity itself \cite{gamifyMotivation}.

\begin{figure}[H]
    \centering
    \includegraphics[width=0.6\linewidth]{images/duolingo_streak.png}
    \caption{Duolingo Streak \cite{duolingoStreak}.}
    \label{fig:duolingo_streak}
\end{figure}

\subsubsection{Attribution Theory}

Attribution theory describes how individuals explain success and failure.
In gamified systems, it is important that users get rewarded for their own effort and strategies rather than to luck or fixed ability.
When implementing effort-based feedback it can increase motivation and encourage continued user engagement.
Additionally, designing penalties and failure in such a way that it only seems temporary and improvable can lead to less frustration, hence making the user more likely to continue using the app \cite{redwhiteAttribution}.

\begin{figure}[H]
    \centering
    \includegraphics[width=\linewidth]{images/attribution_theory_2.png}
    \caption{Attributional Process of Motivation.}
    \label{fig:attribution_theory_2}
\end{figure}

\subsubsection{Expectancy-Value Theory}

Expectancy-value theory explains how what can motivate a user before engaging in an activity, those being: the expectation of success and the subjective value of the task.
When a user thinks that they can succeed in an activity and values it, they are more likely to engage in it.

\begin{enumerate}
    \item \textbf{Expectancy for Success:} 
    Expectance for success can be defined with the question: "How likely is it that I will succeed in this activity?" 
    In games and gamified systems, this belief is answered with a clear definition of goals, transparent rules, the difficulty of the task, and feedback provided during the activity.
    To make the user have a clear vision of their success, it is important to provide clear objectives, transparent mechanics, and visible indicators of progress.
    When the criterias of a "Is this challenge achievable?" question is met, the user is more likely to invest effort in it \cite{scienceDirectExpectancy}.
    
    \item \textbf{Subjective Task Value:}  
    Subjective Task Value, describes if the task is worth the effort, which is subjective for each user.
    This value consists of several components, including intrinsic value, the importance of performing well, usefulness for future goals, and effort, such as time investment.
    In games a value can be increased by making the user progress faster, reducing frustration or making those tasks relevant for future goals.\cite{scienceDirectExpectancy}.
\end{enumerate}

\subsection{Engagement and Habit Formation}

\begin{itemize}
    \item Feedback loops
    \item Short-term versus long-term engagement
    \item Daily routines and behavioral reinforcement
\end{itemize}
