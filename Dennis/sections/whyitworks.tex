%!TEX root = ../../Diplomarbeit.tex
\section{Why Gamification Works for “Normal” Activities}
This chapter answers the main thesis question directly by exploring how gamification transforms everyday activities.

\subsection{Transformation of Mundane Tasks}
\begin{itemize}
    \item \textbf{Turning maintenance into progress:} Routine tasks often feel repetitive and unrewarding. Gamification provides a sense of progression for these necessary actions.
    \item \textbf{Making invisible progress visible:} Many real-world improvements happen too slowly to notice. Game elements quantify this progress immediately.
    \item \textbf{Micro-rewards for delayed real-world outcomes:} While the real-world benefit might take months (e.g., a plant blooming), virtual rewards provide instant gratification.
\end{itemize}

\subsection{Feedback in Real-World Processes}
\begin{itemize}
    \item \textbf{Plants grow slowly vs. App feedback:}  Biological processes are slow; the app bridges this gap with instant feedback loops.
    \item \textbf{Sensors for real-time gratification:} IoT sensors detect actions immediately, allowing the system to reward the user the moment they care for the plant.
    \item \textbf{Visual progress vs. real growth:} Virtual representations can grow and evolve visibly in the app even during periods where the real plant shows little visible change.
\end{itemize}

\subsection{Emotional Attachment \& Ownership}
\begin{itemize}
    \item \textbf{Avatars / Plant Identity:} Giving the plant a digital persona creates a relationship beyond simple object ownership.
    \item \textbf{Personal Investment:} Customization options increase the user's emotional stake in the system (`IKEA effect').
    \item \textbf{Long-term Bonding:} Reviewing history and milestones strengthens the connection over time.
\end{itemize}
