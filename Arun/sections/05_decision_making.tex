\section{Data-Driven Decision Making in Plant Care (approx. 4 Pages)}

\subsection{Using Collected Data for Care Optimization}
\begin{itemize}
    \item Definition of acceptable value ranges for each parameter.
    \item Classification into condition zones:
    \begin{itemize}
        \item Green Zone: Optimal conditions, no action required.
        \item Yellow Zone: Suboptimal conditions, attention recommended.
        \item Red Zone: Critical conditions, immediate action required.
    \end{itemize}
    \item Threshold values depend on plant species and growth stage.
    \item Example differences:
    \begin{itemize}
        \item Cactus: Low soil moisture tolerance, high light preference.
        \item Fern: High humidity requirement, low direct light tolerance.
    \end{itemize}
    \item Thresholds stored as configurable parameters rather than fixed values.
\end{itemize}

\subsection{Deriving Care Recommendations}
\begin{itemize}
    \item Individual sensor values interpreted in context.
    \item Combination of multiple parameters for decision making.
    \item Example rule-based logic:
    \begin{itemize}
        \item High temperature + low soil moisture $\rightarrow$ urgent watering alert.
        \item High humidity + low temperature $\rightarrow$ mold risk warning.
        \item Low light levels over extended periods $\rightarrow$ relocation suggestion.
    \end{itemize}
    \item Recommendations translated into clear, user-friendly actions.
    \item Focus on actionable feedback rather than raw data presentation.
\end{itemize}

\subsection{Visualization and Interpretation}
\begin{itemize}
    \item Visual representation simplifies complex sensor data.
    \item Traffic light system (Green / Yellow / Red) for quick status recognition.
    \item Graphical trend lines show changes over time.
    \item Enables users to identify patterns rather than isolated values.
    \item Historical comparison supports long-term care decisions.
\end{itemize}

\subsection{Automated Systems}
\begin{itemize}
    \item Concept of closed-loop systems in plant care.
    \item Sensor data used to trigger automated actions.
    \item Example: automatic activation of a water pump when soil moisture drops below a defined threshold.
    \item Automation considered as a future extension of the system.
    \item Emphasis on maintaining user control and safety.
\end{itemize}

\subsection*{Possible Sources for Data-Driven Decision Making in Plant Care}

\begin{itemize}
    \item Rule-based decision support systems in agriculture:  
    \url{https://pmc.ncbi.nlm.nih.gov/articles/PMC9570642/}

    \item Effectiveness of decision support systems in plant protection:  
    \url{https://www.nature.com/articles/s43247-021-00291-8}

    \item Optimal environmental ranges for indoor plants:  
    \url{https://help.mygardyn.com/en/articles/1777089}

    \item Light intensity requirements for plant growth:  
    \url{https://consensus.app/questions/range-light-intensity-plant-need-grow-well/}

    \item Closed-loop and automated irrigation concepts:  
    \url{https://sensorex.com/how-does-an-irrigation-control-system-work/}

    \item Feedback loop principles in irrigation control systems:  
    \url{https://www.irrigation.org/IA/FileUploads/IA/Resources/TechnicalPapers/2009/IrrigationControlWithFeedbackLoops.pdf}

    \item IoT data visualization principles for sensor-based systems:  
    \url{https://www.viam.com/post/harnessing-the-power-of-tableau-to-visualize-sensor-data}

    \item Real-time visualization of time-series sensor data:  
    \url{https://www.influxdata.com/blog/real-time-visualization-iiot-data/}
\end{itemize}
